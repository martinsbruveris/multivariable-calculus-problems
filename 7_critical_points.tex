\begin{question}
True or false: if $(a,b)$ is a critical point of $f(x,y)$, then $(a,b)$ is a local extremum of $f(x,y)$.
\end{question}

\begin{solution}
False.

It is true, that if $(a,b)$ is a local extremum of $f(x,y)$ and the partial derivatives of $f(x,y)$ exist at $(a,b)$, then $(a,b)$ is a critical point of $f(x,y)$. However, the other implication is not true in general, since a critical point can also be a saddle point.
\end{solution}

\begin{question}
True or false: a point $(a,b)$ is a critical point of $f(x,y)$ if $f_x(a,b)=0$ and $f_y(a,b) = 0$.
\end{question}

\begin{solution}
True.

This is the definition of a critical point.
\end{solution}

\begin{question}
True or false: a point $(a,b)$ is a critical point of $f(x,y)$ if $f_x(a,b)$ and $f_y(a,b)$ are undefined.
\end{question}

\begin{solution}
False.

A point is a critical point, if both partial derivatives are defined at the point and equal to $0$.
\end{solution}

\begin{question}
Find the critical points of each of the functions. Decide by inspection whether each of the critical points is a maximum, minimum or a saddle point.
\begin{tasks}(3)
\task
$f(x,y) = x^2+2y^2$
\task
$f(x,y) = x^2-2y^2$
\task
$f(x,y) = e^{-x^2-7y^2+3}$
\end{tasks}
\end{question}

\begin{solution}
\begin{enumerate}
\item
We have
\begin{align*}
f_x &= 2x &
f_y &= 4y\,,
\end{align*}
and hence the only critical point is $(0,0)$. We have $f(0,0) = 0$ and because
\[
f(x,y) = x^2 + 2y^2 \geq 0\,,
\]
we have the inequality $f(x,y) \geq f(0,0)$ for all points $(x,y)$. Thus $(0,0)$ is a global minimum.
\item
We have
\begin{align*}
f_x &= 2x &
f_y &= -4y\,,
\end{align*}
and hence the only critical point is $(0,0)$ with $f(0,0) = 0$. We note however that for $x \neq 0$ we have $f(x,0) =x^2 > 0$, while for $y \neq 0$ we have $f(y,0) = -2y^2 < 0$. In particular $f(1,0) > f(0,0)$, while $f(0,1) < f(0,0)$. Thus $(0,0)$ is neither a maximum nor a minimum; hence it has to be a saddle point.
\item
We have
\begin{align*}
f_x &= -2x e^{-x^2-7y^2+3} &
f_y &= -14y e^{-x^2-7y^2+3}\,,
\end{align*}
and hence the only critical point is $(0,0)$ with $f(0,0) = e^3$. Since $-x^2 -7^2 \leq 0$, we have
\[
f(x,y) = e^{-x^2-7y^2+3} = e^{-x^2-7y^2} \cdot e^3 \leq e^3 = f(0,0)\,.
\]
This implies that the point $(0,0)$ is a global maximum.
\end{enumerate}
\end{solution}

\begin{question}
Use the maximum-minimum test for quadratic functions to decide whether $(0,0)$ is a maximum, minimum or a saddle point.
\begin{tasks}(2)
\task
$f(x,y) = x^2+xy+y^2$
\task
$f(x,y) = y^2-x^2+3xy$
\task
$f(x,y) = 3+2x^2-xy+y^2$
\task
$f(x,y) = 1-x^2+2xy-6y^2$
\end{tasks}
\end{question}

\begin{solution}
Note that the general form of a quadratic function is
\[
g(x,y) = Ax^2 + 2B xy + C y^2\,;
\]
pay particular attention to the factor $2$ in $2B$.
\begin{enumerate}
\item
$A=1$, $B=\frac 12$, $C=1$. We have
$A > 0$ and $AC-B^2= \frac 34 > 0$
and thus $(0,0)$ is a minimum.

\item
$A=-1$, $B=\frac 32$, $C=1$. We have 
$AC-B^2 = -\frac {13}4 < 0$ 
and thus $(0,0)$ is a saddle point.

\item
$A=2$, $B=-\frac 12$, $C=1$. We have 
$A > 0$ and $AC-B^2 = \frac {7}4 > 0$
and thus $(0,0)$ is a minimum.

\item
$A=-1$, $B = 1$, $C=-6$. We have
$A < 0$ and $AC-B^2 = 1 > 0$
and thus $(0,0)$ is a maximum.
\end{enumerate}
\end{solution}

\begin{question}
Explain, why the maximum-minimum test for quadratic functions is a special case of the second derivative test.
\end{question}

\begin{solution}
Consider the quadratic function
\[
g(x,y) = Ax^2 + 2Bxy + Cy^2\,.
\]
The only critical point is $(0,0)$ and the second derivatives of $g(x,y)$ are
\begin{align*}
g_{xx} &= 2A &
g_{xy} &= 2B &
g_{yy} &= 2C\,.
\end{align*}
The second derivative test states that if at $(0,0)$ we have $g_{xx} < 0$ and $g_{xx}g_{yy} - g_{xy}^2 > 0$, then $(0,0)$ is a local minimum. For our function $g(x,y)$ this means
\[
2A < 0 \text{ and } 4AC - 4B^2 > 0\,.
\]
When we divive the first inequality by $2$ and the second one by $4$, we recover the maximum-minimum test for quadratic functions. The same argument can be used for the other types of critical points.

\begin{note*}
There is one slight difference between the two tests though. The second derivative test tells us when a critical point is a \emph{local} maximum or minimum. For a quadratic function it is true that a local maximum or minimum is in fact a \emph{global} one.
\end{note*}
\end{solution}

\begin{question}
Find the critical points of the following functions and classify them using the second derivative test.
\begin{tasks}(2)
\task
$f(x,y) = 2x^2-2xy+y^2-2x+1$
\task
$f(x,y) = x^3 + 3xy + y^3$
\task
$f(x,y) = 5 + 3x^2 + x^3 + 3y^2$
\task
$f(x,y) = x^3 - 2x^2 - 4y^2$
\task
$f(x,y) = -1 + 2x^2 - \frac 12 y^2 + y^4 + 2x^3$
\task
$f(x,y) = \frac 1x + xy + \frac 1y$
\end{tasks}
\end{question}

\begin{solution}
\begin{enumerate}
\item
The partial derivatives of $f(x,y)$ are
\begin{align*}
f_x(x,y) &= 4x - 2y - 2 &
f_y(x,y) &= -2x + 2y\,.
\end{align*}
Critical point are solutions to the equations
\begin{align*}
4x - 2y - 2 &= 0 \\
-2x + 2y &= 0\,.
\end{align*}
The second equation gives $x=y$ and the first equation leads to $y=1$. Thus the only critical point is $(1,1)$. To determine the type of the critical point we compute
\begin{align*}
f_{xx}(x,y) &= 4 &
f_{xy}(x,y) &= -2 &
f_{yy}(x,y) &= 2\,,
\end{align*}
and so
\begin{center}
%\renewcommand{\arraystretch}{1.25}
\begin{tabular}{c|ccc|cc}
 & $A$ & $B$ & $C$ & $AC-B^2$ & Type \\ \hline
$\left(1, 1\right)$ & $4$ & $-2$ & $2$ & $4$ & Local minimum
\end{tabular}
\end{center}

\item
The partial derivatives of $f(x,y)$ are
\begin{align*}
f_x &= 3x^2+3y &
f_y &= 3x+3y^2\,.
\end{align*}
Critical points are solutions to the equations
\begin{align*}
x^2+y &= 0 \\
x+y^2 &=0\,.
\end{align*}
We use the first equation to substitute $y=-x^2$ into the second equation to obtain
\[
x+x^4 = 0\,.
\]
This can be factorized into $x(1+x^3) =0$ leading to two solutions $x=0$ and $x=-1$. Thus we obtain the two critical points $(0,0)$ and $(-1,-1)$. To determine their type we compute
\begin{align*}
f_{xx} &= 6x &
f_{xy} &= 3 &
f_{yy} &= 6y\,.
\end{align*}
Evaluated at the critical points we get
\begin{center}
%\renewcommand{\arraystretch}{1.25}
\begin{tabular}{c|ccc|cc}
 & $A$ & $B$ & $C$ & $AC-B^2$ & Type \\ \hline
$\left(0, 0\right)$ & $0$ & $3$ & $0$ & $-9$ & Saddle point \\
$\left(-1, -1\right)$ & $6$ & $3$ & $6$ & $27$ & Local minimum
\end{tabular}
\end{center}

\item
The partial derivatives of $f(x,y)$ are
\begin{align*}
f_x &= 6x + 3x^2 &
f_y &= 6y\,.
\end{align*}
Critical points are solutions of
\begin{align*}
2x + x^2 &= 0 \\
y &=0\,.
\end{align*}
We have two critical points at $(0,0)$ and $(-2,0)$. To determine their type we compute
\begin{align*}
f_{xx} &= 6+6x &
f_{xy} &= 0 &
f_{yy} &= 6 \,.
\end{align*}
Thus we have
\begin{center}
%\renewcommand{\arraystretch}{1.25}
\begin{tabular}{c|ccc|cc}
 & $A$ & $B$ & $C$ & $AC-B^2$ & Type \\ \hline
$\left(0, 0\right)$ & $6$ & $0$ & $6$ & $36$ & Local minimum \\
$\left(-2, 0 \right)$ & $-6$ & $0$ & $6$ & $-36$ &  Saddle point \\
\end{tabular}
\end{center}

\item
The partial derivatives of $f(x,y)$ are
\begin{align*}
f_x &= 3x^2 - 4x &
f_y &= -8y\,.
\end{align*}
Critical points are solutions of
\begin{align*}
3x^2 -4x &= 0 \\
y &=0\,.
\end{align*}
We have two critical points at $(0,0)$ and $\left(\frac 43,0\right)$. To determine their type we compute
\begin{align*}
f_{xx} &= 6x-4 &
f_{xy} &= 0 &
f_{yy} &= -8 \,.
\end{align*}
Thus we have
\begin{center}
%\renewcommand{\arraystretch}{1.25}
\begin{tabular}{c|ccc|cc}
 & $A$ & $B$ & $C$ & $AC-B^2$ & Type \\ \hline
$\left(0, 0\right)$ & $-4$ & $0$ & $-8$ & $32$ & Local maximum \\
$\left(4/3, 0 \right)$ & $4$ & $0$ & $-8$ & $-32$ &  Saddle point \\
\end{tabular}
\end{center}

\item
The partial derivatives of $f(x,y)$ are
\begin{align*}
f_x &= 4x + 6x^2 &
f_y &= -y + 4y^3\,.
\end{align*}
Critical points are solutions of
\[
\begin{aligned}
2x + 3x^2 &= 0 \\
-y + 4y^3 &=0
\end{aligned}
\quad\Leftrightarrow\quad
\begin{aligned}
x(2+3x) &= 0 \\
y(4y^2-1) &= 0\,.
\end{aligned}
\]
We have six solutions $(0,0)$, $\left(0,\frac 12\right)$, $\left(0,-\frac 12\right)$, $\left(-\frac 23,0\right)$, $\left(-\frac 23,\frac 12\right)$, $\left(-\frac 23,-\frac 12\right)$.
To determine their type we compute
\begin{align*}
f_{xx} &= 4 + 12x &
f_{xy} &= 0 &
f_{yy} &= -1 + 12y^2\,.
\end{align*}
Thus we have
\begin{center}
%\renewcommand{\arraystretch}{1.25}
\begin{tabular}{c|ccc|cc}
 & $A$ & $B$ & $C$ & $AC-B^2$ & Type \\ \hline
$\left(0, 0\right)$ & $4$ & $0$ & $-1$ & $-4$ &  Saddle point \\
$\left(0, \pm 1/2 \right)$ & $4$ & $0$ & $2$ & $8$ & Local minimum \\
$\left(-2/3, 0 \right)$ & $-4$ & $0$ & $-1$ & $4$ & Local maximum\\
$\left(-2/3, \pm 1/2 \right)$ & $-4$ & $0$ & $2$ & $-8$ & Saddle point
\end{tabular}
\end{center}

\item
The partial derivatives of $f(x,y)$ are
\begin{align*}
f_x &= -\frac 1{x^2} + y & 
f_y &= x - \frac{1}{y^2}\,.
\end{align*}
Critical points are solutions of
\begin{align*}
-\frac 1{x^2} + y &= 0 \\
x - \frac{1}{y^2} &=0\,.
\end{align*}
The only critical point is $(1,1)$. To determine its type we compute
\begin{align*}
f_{xx} &= \frac{2}{x^3} &
f_{xy} &= 1 &
f_{yy} &= \frac{2}{y^3}\,.
\end{align*}
Evaluated at $(1,1)$ we get
\begin{center}
%\renewcommand{\arraystretch}{1.25}
\begin{tabular}{c|ccc|cc}
 & $A$ & $B$ & $C$ & $AC-B^2$ & Type \\ \hline
$\left(1, 1 \right)$ & $2$ & $1$ & $2$ & $3$ & Local minimum
\end{tabular}
\end{center}
\end{enumerate}
\end{solution}

\begin{question}
Find and classify the critical points of the following functions.
\begin{tasks}(2)
\task
$f(x,y) = x^3+3xy^2-15x+y^3-15y$
\task
$f(x,y) = (x^2+y^2)e^{x^2-y^2}$
\task
$f(x,y) = \ln(ax^2 + by^2 + 1)$ with $a,b > 0$
\task
$f(x,y) = \frac{x^3-3x}{1+y^2}$
\end{tasks}
\end{question}

\begin{solution}
\begin{enumerate}
\item
The partial derivatives of $f(x,y)$ are
\begin{align*}
f_x(x,y) &= 3x^2+3y^2-15 &
f_y(x,y) = 6xy+3y^2-15 \,.
\end{align*}
Critical point are solutions to the equations
\begin{align*}
3x^2+3y^2-15 &= 0 \\
6xy+3y^2-15 &= 0\,.
\end{align*}
We subtract the second equation from the first to obtain
\[
3x^2 - 6xy = 0
\quad\Leftrightarrow\quad
3x \left(x-2y\right) = 0\,.
\]
We have two cases two consider.
\begin{itemize}
\item
If $x=0$, then the first equation gives
\[
3y^2 - 15 = 0\;\Rightarrow\; y = \pm \sqrt{5}\,,
\]
leading to the two critical points $\left(0, \sqrt 5\right)$ and $\left(0, -\sqrt 5\right)$.
\item
If $x=2y$, then the first equation becomes
\[
15y^2 = 15\;\Rightarrow\; y = \pm 1\,,
\]
leading to two more critical points $(1, 2)$ and $-1, -2)$.
\end{itemize}
To determine the type of the critical points we need the second derivatives
\begin{align*}
f_{xx}(x,y) &= 6x &
f_{xy}(x,y) &= 6y &
f_{yy}(x,y) &= 6x+6y\,.
\end{align*}
Thus we have
\begin{center}
\renewcommand{\arraystretch}{1.25}
\begin{tabular}{c|ccc|cc}
 & $A$ & $B$ & $C$ & $AC-B^2$ & Type \\ \hline
$\left(0, \sqrt 5\right)$ & $0$ & $6\sqrt{5}$ & $6\sqrt{5}$ & $-180$ & Saddle point \\
$\left(0, -\sqrt 5\right)$ &  $0$ & $-6\sqrt{5}$ & $-6\sqrt{5}$ & $-180$ & Saddle point \\
$\left(2,1 \right)$ &  $12$ & $6$ & $18$ & $180$ & Local minimum \\
$\left(-2,-1 \right)$ &  $-12$ & $-6$ & $-18$ & $180$ & Local maxmimum \\
\end{tabular}
\end{center}

\item
The partial derivatives of $f(x,y)$ are
\begin{align*}
f_x(x,y) &= e^{x^2-y^2} \left( 2x + 2x\left(x^2 + y^2\right)\right) &
f_y(x,y) &= e^{x^2-y^2} \left( 2y - 2y\left(x^2 + y^2\right)\right) \,.
\end{align*}
Critical point are solutions to the equations
\begin{align*}
e^{x^2-y^2} \left( 2x + 2x\left(x^2 + y^2\right)\right) &= 0\\
e^{x^2-y^2} \left( 2y - 2y\left(x^2 + y^2\right)\right) &= 0 \,.
\end{align*}
This system is equivalent to
\begin{align*}
2x\left(x^2 + y^2 + 1 \right) &= 0\\
2y\left(x^2 + y^2 - 1 \right) &= 0 \,.
\end{align*}
In the first equation we see that $x^2 + y^2 + 1 \neq 0$ and so $x=0$. From the second equation we obtain
\[
y=0 \text{ or } x^2 + y^2 = 1\,.
\]
This leads to the three solutions $(0,0)$, $(0,1)$ and $(0,-1)$.

To determine the type of the critical points we need the second derivatives
\begin{align*}
f_{xx}(x,y) &= e^{x^2-y^2}\left(2 + 6x^2 + 2y^2 + 4x^2 + 4x^2\left(x^2 + y^2\right)\right) \\
f_{xy}(x,y) &= e^{x^2-y^2}\left(4xy - 4xy - 4xy\left(x^2 + y^2\right)\right) \\
f_{yy}(x,y) &= e^{x^2-y^2}\left(2 - 2x^2 - 6y^2 - 4y^2 + 4y^2 \left(x^2 + y^2\right)\right)
\end{align*}
Thus we have
\begin{center}
\renewcommand{\arraystretch}{1.25}
\begin{tabular}{c|ccc|cc}
 & $A$ & $B$ & $C$ & $AC-B^2$ & Type \\ \hline
$\left(0, 0\right)$ & $2$ & $0$ & $2$ & $4$ & Local minimum \\
$\left(0, 1\right)$ &  $4e^{-1}$ & $0$ & $-4e^{-1}$ & $-16e^{-2}$ & Saddle point \\
$\left(0, -1\right)$ &  $4e^{-1}$ & $0$ & $-4e^{-1}$ & $-16e^{-2}$ & Saddle point
\end{tabular}
\end{center}

\item
The partial derivatives of $f(x,y)$ are
\begin{align*}
f_x(x,y) &= \frac{2ax}{ax^2 + by^2 + 1} &
f_y(x,y) &= \frac{2by}{ax^2 + by^2 + 1} \,.
\end{align*}
Critical point are solutions to the equations
\begin{align*}
\frac{2ax}{ax^2 + by^2 + 1} &= 0 \\
\frac{2ay}{ax^2 + by^2 + 1} &= 0\,.
\end{align*}
Because $ax^2 + by^2 + 1 \neq 0$, the only critical point is $(0,0)$. To determine the type of the critical point we need the second derivatives
\begin{align*}
f_{xx}(x,y) &= \frac{2a}{ax^2 + by^2 + 1} - \frac{4a^2x^2}{\left(ax^2 + by^2 + 1\right)^2}\\
f_{xy}(x,y) &= - \frac{4abxy}{\left(ax^2 + by^2 + 1\right)^2}\\
f_{yy}(x,y) &= \frac{2b}{ax^2 + by^2 + 1} - \frac{4b^2y^2}{\left(ax^2 + by^2 + 1\right)^2}
\end{align*}
Thus we have
\begin{center}
\renewcommand{\arraystretch}{1.25}
\begin{tabular}{c|ccc|cc}
 & $A$ & $B$ & $C$ & $AC-B^2$ & Type \\ \hline
$\left(0, 0\right)$ & $2a$ & $0$ & $2b$ & $4ab > 0$ & Local minimum
\end{tabular}
\end{center}

\item
The partial derivatives of $f(x,y)$ are
\begin{align*}
f_x(x,y) &= \frac{3x^2-3}{1+y^2} &
f_y(x,y) &= -2y \frac{x^3-3x}{\left(1+y^2\right)^2} \,.
\end{align*}
Critical point are solutions to the equations
\begin{align*}
\frac{3x^2-3}{1+y^2} &= 0 \\
-2y \frac{x^3-3x}{\left(1+y^2\right)^2} &= 0\,.
\end{align*}
We can multiply both equations by $(1+y^2)$ and $\left(1+y^2\right)^2$ respectively to obtain
\begin{align*}
3\left(x^2 - 1\right) &= 0\;\Rightarrow\; x = \pm 1 \\
-2xy\left(x^2 - 3 \right) &= 0\;\Rightarrow\; y = 0 \text{ or }x = 0 \text{ or } x = \pm \sqrt{3}\,.
\end{align*}
Note however that since the first equation specifies $x= \pm 1$, it follows that we have $y=0$, because the other cases in the second equation cannot happen. Thus we have the two critical points $(1,0)$ and $(-1,0)$. To determine the type of the critical points we need the second derivatives
\begin{align*}
f_{xx}(x,y) &= \frac{6x}{1+y^2}\\
f_{xy}(x,y) &= -2y \frac{3x-3}{\left(1 + y^2\right)^2}\\
f_{yy}(x,y) &= -2\frac{x^3-3x}{\left(1+y^2\right)^2} +8y^2 \frac{x^3-3x}{\left(1+y^2\right)^3}
\end{align*}
Thus we have
\begin{center}
\renewcommand{\arraystretch}{1.25}
\begin{tabular}{c|ccc|cc}
 & $A$ & $B$ & $C$ & $AC-B^2$ & Type \\ \hline
$\left(1, 0\right)$ & $6$ & $0$ & $4$ & $24$ & Local minimum \\
$\left(-1, 0\right)$ & $-6$ & $0$ & $-4$ & $24$ & Local maximum
\end{tabular}
\end{center}
\end{enumerate}
\end{solution}

\begin{question}
Find the point on the plane $x+2y+3z-10=0$, that minimizes the distance
\begin{tasks}(2)
\task
to the origin;
\task
to the point $(1,1,1)$.
\end{tasks}

\begin{hint*}
Minimize the squared distance to simplify calculations.
\end{hint*}
\end{question}

\begin{solution}
\begin{enumerate}
\item
It is easiest to express each point on the plane in the form
\[
(10-2y-3z, y, z) \text{ with } (y,z) \text{ a point in the $yz$-plane.}
\]
The function we want to minimize is the squared distance
\[
f(y,z) = (10-2y-3z)^2 + y^2 + z^2\,.
\]
Critical points are solutions of
\begin{align*}
2y - 4(10-2y-3z) &= 0 \\
2z - 6 (10-2y-3z) &= 0\,.
\end{align*}
The only solution is $\displaystyle y=\frac{10}7$, $\displaystyle z=\frac{15}7$. The corresponding $x$-coordinate is
\[
x = 10-2\frac {10}7 -3\frac{15}7 = \frac 5 7\,.
\]
Hence the point on the plane, closest to the origin, is $\displaystyle \left(\frac 57, \frac{10}7, \frac{15}7 \right)$.

\item
In this case the function we want to minimize is
\[
f(y,z) = (10-2y-3z-1)^2 + (y-1)^2 + (z-1)^2\,.
\]
Critical points are solutions of
\begin{align*}
2y-2 - 4(9-2y-3z) &= 0 \\
2z-2 - 6 (9-2y-3z) &= 0\,.
\end{align*}
The only solution is $\displaystyle y=\frac{11}7$, $\displaystyle z=\frac{13}7$. The corresponding $x$-coordinate is
\[
x = 10-2\frac {11}7 -3\frac{13}7 = \frac 9 7\,.
\]
Hence the point on the plane, closest to $(1,1,1)$, is $\displaystyle \left(\frac 97, \frac{11}7, \frac{13}7 \right)$.
\end{enumerate}
\end{solution}

\begin{question}
Let $f(x,y) > 0$ for all $x$, $y$. Show that the functions $f(x,y)$ and $g(x,y) = [f(x,y)]^2$ have the same critical points with the same type (local maximum, minimum or saddle point).
\end{question}

\begin{solution}
The partial derivatives of $g(x,y)$ are
\begin{align*}
g_x(x,y) &= 2f(x,y)f_x(x,y) &
g_y(x,y) &= 2f(x,y)f_y(x,y) \,.
\end{align*}
The critical points of $g(x,y)$ are solutions of
\begin{align*}
2f(x,y)f_x(x,y) &= 0 \\
2f(x,y)f_y(x,y) &= 0\,.
\end{align*}
Since we assume that $f(x,y) > 0$, we can divide both equations by $2f(x,y)$. Thus $(a,b)$ is a solution to the above system of equations if and only if $f_x(a,b) = 0$ and $f_y(a,b) =0$. In other words, $(a,b)$ is a critical point of $g(x,y)$ if and only if it is a critical point of $f(x,y)$.

Let us consider the type of the critical point. The second partial derivatives of $g(x,y)$ are
\begin{align*}
g_{xx}(x,y) &= 2f_x(x,y)^2 + 2f(x,y)f_{xx}(x,y) \\
g_{xy}(x,y) &= 2f_x(x,y)f_y(x,y) + 2f(x,y)f_{xy}(x,y) \\
g_{yy}(x,y) &= 2f_y(x,y)^2 + 2f(x,y)f_{yy}(x,y)\,.
\end{align*}
However, at a critical point $(a,b)$ they simplify to
\begin{align*}
g_{xx}(a,b) &= 2f(a,b)f_{xx}(a,b) &
g_{xx}(a,b) &= 2f(a,b)f_{xy}(a,b) &
g_{xx}(a,b) &= 2f(a,b)f_{yy}(a,b)\,.
\end{align*}
Because $f(a,b) > 0$, we see that $g_{xx}(a,b)$ and $f_{xx}(a,b)$ have the same sign, and because at a critical point
\[
g_{xx}g_{yy} - g_{xy}^2 = 4f^2 (f_{xx}f_{yy} - f_{xy}^2)\,,
\]
also the signs of $g_{xx}g_{yy} - g_{xy}^2$ and $f_{xx}f_{yy} - f_{xy}^2$ coincide at $(a,b)$. Therefore the type of $(a,b)$ as a critical point of $f(x,y)$ is the same as a critical point of $g(x,y)$.

\begin{note*}
How would the result change, if we assumed $f(x,y) < 0$ for all $x$, $y$ instead?
\end{note*}
\end{solution}

\begin{question}
For which values of $k$ does
\[
f(x,y) = x^2 + kxy + 4y^2
\]
have a local minimum at $(0,0)$?
\end{question}

\begin{solution}
To study the type of a critical point we look at second derivatives.
\begin{align*}
f_{xx} &= 2 &
f_{xy} &= k &
f_{yy} &= 8\,.
\end{align*}
We see that $(0,0)$ is a minimum, when $16-k^2 > 0$ or equivalently $|k| < 4$. When $|k| > 4$, the point $(0,0)$ is a saddle point.

What happens for $k=\pm 4$? Then the second derivative test is inconclusive and we have to take a closer look at the function. We have
\[
f(x,y) = x^2 \pm 4xy + 4y^2 = (x \pm 2y)^2 \geq 0\,.
\]
Thus $(0,0)$ is a minimum for $k =\pm 4$.
\end{solution}

\begin{question}
Find two numbers $a$ and $b$, such that
\[
\int_a^b (6-x-x^2) \ud x
\]
has its largest value.
\end{question}

\begin{solution}
The problem asks us to find the interval $[a,b]$, that maximizes the value of the integral
\begin{wrapfigure}{r}{0.35\textwidth}
    \centering
\begin{tikzpicture}[scale=0.46875, baseline=(X.base)]
 \def\scale{0.46875}

 \draw[coordinate grid, step=1] (-4, -4) grid (4, 4);
 \node at (0,-2) (X) {};
 \drawaxes{-4}{-4}{4}{4}

 \clip (-4, -4) rectangle (4, 4);

 \fill[pattern=north east lines, pattern color=blue!30] 
   (-2, 0) -- (-2, 2)
   -- plot[parametric, domain={-2:2}, samples=100] 
        function {t, 0.5*(6-t-t**2)}
   -- (-2, 0);

 \fill[pattern=north east lines, pattern color=red!30] 
   (2, 0) -- (3, 0) -- (3,-3)
   -- plot[parametric, domain={3:2}, samples=100] 
        function {t, 0.5*(6-t-t**2)} -- (2,0);

 \draw[draw=black, domain=-4:4, samples=100]
      plot[parametric] function {t, 0.5*(6-t-t**2)};

 \drawxlabels[]{-3/-3, -2/a, 2/2, 3/b}
 \drawylabels[]{3/6}

 \node at (-0.5,1.5) {+};
 \node at (2.75,-1.25) {--};
\end{tikzpicture}
\end{wrapfigure}
\[
\int_a^b (6-x-x^2) \ud x\,.
\]
Because the integral computes the \emph{signed} area between the graph of a function and the $x$-axis, it is clear from the figure that the integral is maximal for $a=-3$ and $b=2$, which are the two roots of the polynomial $6-x-x^2$.

We can check our result using multivariable calculus. Consider the function
\[
F(a,b) = \int_a^b (6-x-x^2) \ud x\,.
\]
Its partial derivatives are
\begin{align*}
F_a(a,b) &= -(6-a-a^2) &
F_b(a,b) &= 6-b-b^2\,,
\end{align*}
and we see that $(-3,2)$ is indeed a critical point of $F(a,b)$.
\end{solution}

\begin{question}
The discriminant $f_{xx}f_{yy} - f_{xy}^2$ is $0$ at the origin for each of the following functions, so the second derivative test fails to determine the type of the critical point. Determine whether the function has a local maximum, minimum, or neither at the origin by imagining what the surface $z=f(x,y)$ looks like. Briefly justify your reasoning in each case.
\begin{tasks}(2)
\task
$f(x,y) = x^2y^2$
\task
$f(x,y) = 1 - x^2y^2$
\task
$f(x,y) = xy^2$
\task
$f(x,y) = x^3y^3$
\end{tasks}
\end{question}

\begin{solution}
\begin{enumerate}
\item
We have $f(0,0) = 0$ and $f(x,y) = x^2y^2 \geq 0$ for all $x$, $y$. Therefore $(0,0)$ is a minimum.
\item
We have $f(0,0) = 1$ and 
\[
f(x,y) = 1 - x^2y^2 \leq 1
\]
for all $x$, $y$. Therefore $(0,0)$ is a maximum.
\item
We have $f(0,0) = 0$. If $x > 0$ and $y \neq 0$, then $f(x,y) > 0$, while for $x < 0$ and $y \neq 0$ we have $f(x,y) < 0$. Thus $(0,0)$ is a saddle point.
\item
We have $f(0,0) = 0$. If $x > 0$ and $y > 0$, then $f(x,y) > 0$, while for $x > 0$ and $y < 0$ we have $f(x,y) < 0$. Thus $(0,0)$ is a saddle point.
\end{enumerate}
\end{solution}


\begin{question}
Consider the general problem of finding the points on a graph $z = g(x,y)$ closest to a given point $(a,b,c)$. Show that $(x_0, y_0)$ is a critical point for the distance from $(x,y,g(x,y))$ to $(a,b,c)$ if and only if the line from $(a,b,c)$ to $(x_0,y_0,g(x_0,y_0))$ is orthogonal to the graph at $(x_0,y_0,g(x_0,y_0))$.
\end{question}

\begin{solution}
Write the squared distance function
\[
d(x,y) = (x-a)^2 + (y-b)^2 + \left(g(x,y) - c\right)^2\,.
\]
Critical points are solutions to the equations
\begin{align*}
x-a + \left(g(x,y) - c\right) g_x(x,y) &= 0 \\
y-b + \left(g(x,y) - c\right) g_y(x,y) &= 0\,.
\end{align*}
Assume $(x_0,y_0)$ is a critical point of $d(x,y)$. The line from $(a,b,c)$ to $(x_0,y_0,g(x_0,y_0))$ can be parametrized by
\[
\ell(t) = \begin{pmatrix}
a \\ b \\ c \end{pmatrix}
+ t \begin{pmatrix}
x_0 - a \\ y_0 - b \\ g(x_0, y_0) - c \end{pmatrix}\,.
\]
As $(x_0, y_0)$ is a critical point, we can rewrite the direction vector of the line $\ell$ in the following way
\[
\begin{pmatrix} x_0 - a \\ y_0 - b \\ g(x_0, y_0) - c \end{pmatrix}
= -\left(g(x_0,y_0) - c\right) 
\begin{pmatrix} g_x(x_0,y_0) \\ g_y(x_0,y_0) \\ -1 \end{pmatrix}\,.
\]
We recognise the vector on the right hand side as a normal vector to the graph of $g(x,y)$ at the point $(x_0,y_0)$. In other words, the line $\ell(t)$ is orthogonal to the graph of $g(x,y)$.

There is one special case to be considered: What if $g(x_0,y_0) - c = 0$? If $(x_0,y_0)$ is a critical point of $d(x,y)$ and $g(x_0,y_0) = c$, then the equations for a critical point imply also $x_0 = a$ and $y_0 = b$; in other words, the points $(a,b,c)$ and $(x_0,y_0,g(x_0,y_0))$ coincide. But then $(a,b,c)$ already lies on the graph of $g(x,y)$ and the distance minimization problem is trivial.
\end{solution}

% \begin{figure}
% \begin{center}
% \includegraphics[width=.48\textwidth]{figures/problems10_4b.png}
% \includegraphics[width=.48\textwidth]{figures/problems10_4c.png}
% \end{center}
% \caption{Contour plot of the function in 4b and a surface plot of the function in 4c.}
% \end{figure}

% \begin{figure}
% \begin{center}
% \includegraphics[width=.48\textwidth]{figures/problems10_5.png}
% \end{center}
% \caption{Surface plot of the function in 5.}
% \end{figure}

%%% Local Variables:
%%% TeX-master: "problems"
%%% End:
