\begin{question}
Find the extreme values of the function $f(x,y)$ on the disk $x^2 + y^2 \leq 1$ without using the method of Lagrange multipliers.
\begin{tasks}(2)
\task
$f(x,y) = 2x^2 + 3y^2$.
\task
$f(x,y) = xy + 5y$.
{\itshape (Requires a calculator)}
\end{tasks}
\end{question}

\begin{solution}
\begin{enumerate}
\item
We first consider the function on the interiour of the disc. Since
\begin{align*}
f_x(x,y) &= 4x &
f_y(x,y) &= 6y\,,
\end{align*}
the only critical point of $f(x,y)$ is $(0,0)$ and $f(0,0) = 0$. Next we consider the function on the boundary $x^2+y^2 = 1$. We parametrize the boundary via
\[
\vec \si(t) = \left( \cos t, \sin t \right)\,,
\]
and consider the composition
\begin{align*}
g(t) &= f(\vec\si(t))
= 2\cos^2 t + 3\sin^2 t
= 2 + \sin^2 t\,.
\end{align*}
Since
\[
g'(t) = 2\sin t \cos t\,,
\]
the critical points of $g(t)$ are $0, \displaystyle \frac \pi 2, \pi, \frac{3\pi}2, 2\pi, \dots$. They correspond to the points
\begin{align*}
(1,0) && (0,1) && (-1,0) && (0,-1)
\end{align*}
on the circle and
\begin{align*}
f(1,0) &= 2 &
f(0,1) &= 3 &
f(-1,0) &= 2 &
f(0,-1) &= 3\,.
\end{align*}
Thus the minimum of $f(x,y)$ is $0$, attained at $(0,0)$ and the maximum is $3$, attained at the two points $(0,\pm 1)$.

\item
We first consider the function on the interiour of the disc. Since
\begin{align*}
f_x(x,y) &= y &
f_y(x,y) &= x+5\,,
\end{align*}
the only critical point of $f(x,y)$ is $(0,-5)$, which lies outside the disc. Next we consider the function on the boundary $x^2+y^2 = 1$. We parametrize the boundary via
\[
\vec \si(t) = \left( \cos t, \sin t \right)\,,
\]
and consider the composition
\begin{align*}
g(t) &= f(\vec \si(t)) = \sin t \cos t + 5\sin t\,.
\end{align*}
Since
\[
g'(t) = \cos^2 t - \sin^2 t + 5 \cos t\,,
\]
critical points are solutions of the equation
\begin{align*}
\cos^2t - \sin^2 t + 5 \cos t &= 0 \\
\cos^2 t + \frac 52 \cos t - \frac 12 &= 0
\quad\Leftrightarrow\quad
\cos t = -\frac 54 \pm \frac{\sqrt{33}}4\,.
\end{align*}
Since $\displaystyle -\frac 54 - \frac{\sqrt{33}}4 < -1$, we only need to consider the other solution. If
\[
\cos t = -\frac 54 + \frac{\sqrt{33}}4 = 0.1861\,,
\]
then
\[
\sin t = \pm \sqrt{1 - \cos^2 t} = \pm 0.9825\,,
\]
and thus
\[
f(0.1861, \pm 0.9825) = \pm 5.0955\,.
\]
Thus the minimum of $f(x,y)$ is $-5.0955$ and the maximum is $5.0955$, each attained at one point.
\end{enumerate}
\end{solution}

\begin{question}
Find the extrema of the function $f(x,y)$ subject to the stated constraints using the method of Lagrange multipliers.
\begin{tasks}(2)
\task
$f(x,y) = x+y$; $x^2+y^2=1$.
\task
$f(x,y) = \cos^2 x + \cos^2 y$; $\displaystyle x+y = \frac \pi 4$.
\task
$f(x,y) = x^2 + y^2$; $x^4 + y^4 = 2$.
\task
$f(x,y) = e^{-xy}$; $x^2 + 4y^2 = 1$.
\end{tasks}
\end{question}

\begin{solution}
\begin{enumerate}
\item
Using the method of Lagrange multipliers, the equations for critical points are
\begin{align*}
1 &= 2x\la &
1 &= 2y\la &
x^2 + y^2 &= 1\,.
\end{align*}
From the first two equations we obtain -- note that none of the variables can equal $0$ --
\begin{align*}
x &= \frac 1{2\la} &
y &= \frac{1}{2\la}\,,
\end{align*}
which we substitute into the third equation,
\begin{align*}
\frac{1}{4\la^2} + \frac{1}{4\la^2} & = 1\\
\la^2 &= \frac 12 \quad\Leftrightarrow\quad
\la = \pm \frac{\sqrt{2}}2\,.
\end{align*}
This leads to the two solutions $\left(\frac{\sqrt{2}}2, \frac{\sqrt{2}}2\right)$ and
$\left(-\frac{\sqrt{2}}2, -\frac{\sqrt{2}}2\right)$. To find the minimum and maximum of $f(x,y)$, we evaluate the function
\begin{align*}
f\left(\frac{\sqrt{2}}2, \frac{\sqrt{2}}2\right) &= \sqrt 2 &
f\left(-\frac{\sqrt{2}}2, -\frac{\sqrt{2}}2\right) &= -\sqrt 2\,.
\end{align*}
Thus the minimum is $-\sqrt{2}$ and the maximum is $\sqrt{2}$.

\item
Using the method of Lagrange multipliers, the equations for critical points are
\begin{align*}
-2 \cos x \sin x &= \la &
-2 \cos y \sin y &= \la &
x + y &= \frac \pi 4\,.
\end{align*}
From the first two equations we obtain 
\begin{align*}
-2 \cos x \sin x &= -2 \cos y \sin y \\
\sin 2x &= \sin 2y\,.
\end{align*}
It follows, using the constraint equation, that
\begin{align*}
\sin 2x = \sin 2\left(\frac \pi 4 - x \right) \quad\Leftrightarrow\quad
\sin 2x &= \sin \left(\frac \pi 2 - 2x \right) \quad\Leftrightarrow\quad
\sin 2x = \cos 2x \\
\tan 2x &= 1\,.
\end{align*}
Thus
\begin{align*}
2x &= \frac \pi 4 + k \pi\,,\text{ with } k \in \mathbb Z\,.
\end{align*}
From this we obtain an infinite number of critical points,
\[
\left(\frac \pi 8 + k \frac \pi 2,
\frac \pi 8 - k \frac \pi 2\right) \,,\text{ with } k \in \mathbb Z\,.
\]
The function $\cos x$ is $2\pi$-periodic, but $\cos^2 x$ has a period of $\pi$, so we only have to consider the values $k=0,1$, when evaluating the function.
\begin{align*}
f\left( \frac \pi 8, \frac \pi 8 \right) &= 2 \cos^2 \frac \pi 8 &
f\left( \frac {5\pi} 8, -\frac {3\pi} 8 \right) &= \cos^2 \frac {5\pi} 8 + \cos^2 \frac {3\pi} 8 = 2 \cos^2\frac {3\pi} 8\,.
\end{align*}
Because $\cos x$ is a decreasing and nonnegative on the interval $\left[0,\frac \pi 2\right]$, it follows that the maximum value is $2 \cos^2 \frac \pi 8$ and the minimum value is $\displaystyle 2 \cos^2 \frac {3\pi} 8$.

\item
Using the method of Lagrange multipliers, the equations for critical points are
\begin{align*}
2x &= 4x^3 \la &
2y &= 4y^3 \la &
x^4 + y^4 &= 2\,.
\end{align*}
The first equation leads to the two cases 
\[
x=0 \quad\text{or}\quad
\frac 12 = x^2 \la\,.
\]

If $x=0$, then the constraint equation simplifies to
\begin{align*}
y^4 &= 2 \quad\Leftrightarrow\quad
y = \pm \sqrt[4]{2}\,,
\end{align*}
and thus we obtain the two solutions $\left(0, \sqrt[4] 2\right)$ and $\left(0, -\sqrt[4] 2\right)$.

If $x \neq 0$, the second equation similarly leads to the two cases
\[
y=0 \quad\text{or}\quad
\frac 12 = y^2 \la\,.
\]
The case $y=0$ gives, via the constraint equation, the two solutions $\left(\sqrt[4] 2, 0\right)$ and $\left(-\sqrt[4] 2, 0\right)$.

Now consider the final case $x \neq 0$ and $y \neq 0$. Then we have
\[
\frac 12 = x^2 \la 
\quad\text{and}\quad
\frac 12 = y^2 \la\,.
\]
We square both equations and add them to obtain
\begin{align*}
\left(x^4 + y^4 \right) \la^2 &= \frac 12 \quad\Leftrightarrow\quad
2 \la^2 = \frac 12 \quad\Leftrightarrow\quad
\la = \pm \frac 12\,.
\end{align*}
When $\la = \displaystyle \frac 12$, we obtain $x^2 = 1$ and $y^2=1$, which gives the four solutions $(1,1)$, $(1,-1)$, $(-1,1)$ and $(-1,-1)$. When $\la = \displaystyle -\frac 12$, we arrive at $x^2 = -1$, which has no solutions. 

Now that we have found eight possible critical points, we shall evaluate the function at them,
\begin{align*}
f\left(0, \pm\sqrt[4] 2\right) &= \sqrt{2} &
f\left(\pm\sqrt[4] 2, 0\right) &= \sqrt{2} &
f(\pm 1, \pm 1) &= 2\,.
\end{align*}
In total we see that the maximum value is $2$, attained at four points and the minimum value is $\sqrt{2}$, also attained at four points.

\item
The Lagrange equations are
\begin{align*}
-y e^{-xy} &= 2x\la &
-x e^{-xy} &= 8y\la &
x^2 + 4y^2 &= 1\,.
\end{align*}
If we multiply the first equation by $x$ and the second one by $y$, we obtain
\[
2x^2\la = 8y^2 \la
\quad\Leftrightarrow\quad
\la(4y^2 - x^2) = 0\,.
\]
Thus either $\la = 0$ or $x^2 = 4y^2$. If $\la = 0$, then the first two equations imply $y=0$ and $x=0$, which contradicts the constraint equation $x^2 + 4y^2 = 1$. Thus we are left with $x^2 = 4y^2$. We substitute this into the constraint equation to obtain $2x^2 = 1$ or $x = \pm \frac{\sqrt 2}{2}$. Next, note that
\[
x^2 = 4y^2
\quad\Leftrightarrow\quad
y = \pm \frac 12 x\,,
\]
and thus we have the four solutions 
$\left( \pm \frac{\sqrt 2}{2}, \pm \frac{\sqrt 2}{4} \right)$, with all four possible sign combinations. We evaluate $f$ at these points
\begin{align*}
f\left( \frac{\sqrt 2}{2}, \frac{\sqrt 2}{4} \right)
= f\left( -\frac{\sqrt 2}{2}, -\frac{\sqrt 2}{4} \right) 
&= e^{-1/4} &
f\left( \frac{\sqrt 2}{2}, -\frac{\sqrt 2}{4} \right)
= f\left( -\frac{\sqrt 2}{2}, \frac{\sqrt 2}{4} \right) 
&= e^{1/4}\,.
\end{align*}
Thus the minimum value of $f$ is $e^{-1/4}$ and the maximum value is $e^{1/4}$, both are attained at two points.
\end{enumerate}
\end{solution}

\begin{question}
Find the points on the sphere $x^2+y^2+z^2=4$, that are closest to and furthest away from the point $(3,3,-1)$.
\end{question}

\begin{solution}
We want to find the minima and maxima of the squared distance function
\[
f(x,y,z) = (x-3)^2 + (y-3)^2 + (z+1)^2\,,
\]
subject to the constraint $g(x,y,z) = 4$, where
$g(x,y,z) = x^2 + y^2 + z^2$.
Using Lagrange multipliers the equations for critical points are
\begin{align*}
2(x-3) &= 2x\la &
2(y-3) &= 2y\la &
2(z+1) &= 2z\la &
x^2 + y^2 + z^2 &= 4\,.
\end{align*}
We rewrite the first three equations as
\begin{align*}
x(1-\la) &= 3 &
y(1-\la) &= 3 &
z(1-\la) &= -1\,.
\end{align*}
Now we multiply the constraint equation by $(1-\la)^2$ to obtain
\begin{align*}
x^2(1-\la)^2 + y^2(1-\la)^2 + z^2(1-\la)^2 &= 4(1-\la)^2 \\
9 + 9 + 1 &= 4(1-\la)^2 \quad\Leftrightarrow\quad
1-\la = \pm \frac{\sqrt{19}}2\,.
\end{align*}
Note that we only need the value of $1-\la$ to find $x$, $y$ and $z$.

When $1-\la = \displaystyle \frac{\sqrt{19}}2$, we obtain
\[
x = \frac 6{\sqrt{19}}\,,\quad
y = \frac 6{\sqrt{19}}\,,\quad
z = -\frac 2{\sqrt{19}}\,,
\]
and when $1-\la = -\displaystyle \frac{\sqrt{19}}2$, we obtain
\[
x = -\frac 6{\sqrt{19}}\,,\quad
y = -\frac 6{\sqrt{19}}\,,\quad
z = \frac 2{\sqrt{19}}\,.
\]
We can check by evaluating the distance function that the point
$\displaystyle\left(\frac 6{\sqrt{19}}, \frac 6{\sqrt{19}}, -\frac 2{\sqrt{19}}\right)$
is the point closest to and
$\displaystyle\left(-\frac 6{\sqrt{19}}, -\frac 6{\sqrt{19}}, \frac 2{\sqrt{19}}\right)$
is the point furthest away from $(3,3,-1)$.
\end{solution}

\begin{question}
 The Baraboo, Sheffield, plant of International Widget Ltd. uses aluminium, iron and magnesium to produce high-quality widgets. The quantity of widgets that may be produced using $x$ tons of aluminium, $y$ tons of iron and $z$ tons of magnesium is $Q(x,y,z) = xyz$. The cost of raw materials is aluminium \pounds6 per ton; iron \pounds4 per ton; and magnesium \pounds8 per ton. How many tons each of aluminium, iron and magnesium should be used to manufacture $1000$ widgets at the lowest possible price?

\begin{hint*}
You want an extreme value for what function? Subject to what constraint?
\end{hint*}
\end{question}

\begin{solution}
We are looking for minima of the cost function
\[
f(x,y,z) = 6x + 4y + 8z
\]
subject to the constraint $Q(x,y,z) = 1000$, where
$Q(x,y,z) = xyz$.
Using Lagrange multipliers the equations for critical points are
\begin{align*}
6 &= yz\la &
4 &= xz \la &
8 &= xy \la &
xyz &= 1000\,.
\end{align*}
We multiply the first equation by $x$, the second by $y$ and the third by $z$ to obtain
\begin{align*}
6x &= xyz \la \quad\Rightarrow\quad x = \frac{500}3 \la \\
4y &= xyz \la \quad\Rightarrow\quad y = 250 \la \\
8z &= xyz \la \quad\Rightarrow\quad z = 125 \la\,.
\end{align*}
We substitute this into the constraint equation to obtain
\begin{align*}
\frac{500}3 \la \cdot 250\la \cdot 125 \la &= 1000
\quad\Leftrightarrow\quad
\la^3 = \frac{3}{25^3} 
\quad\Leftrightarrow\quad
\la = \frac{\sqrt[3]{3}}5\,.
\end{align*}
Hence the solution is
\begin{align*}
x &= \frac{20}3 \sqrt[3]{3} &
y &= 10 \sqrt[3]{3} &
z &= 5 \sqrt[3]{3}\,.
\end{align*}
Therefore the least cost for producing $1000$ widgets is achieved using $\frac{20}3 \sqrt[3]{3}$ tons of aluminium, $\sqrt[3]{3}$ tons of iron and $\sqrt[3]{3}$ tons of magnesium.
\end{solution}

\begin{question}
A firm uses wool and cotton fiber to produce cloth. The amount of cloth produced is given by $Q(x,y) = xy-x-y+1$, where $x$ is the number of pounds of wool, $y$ is the number of pounds of cotton and $x \geq 1$ and $y \geq 1$. If wool costs $p$ dollars per pound, cotton costs $q$ dollars per pound, and the firm can spend $B$ dollars on material, what should the mix of cotton and wool be to produce the most cloth?
\end{question}

\begin{solution}
We are looking for maximum of the function
\[
Q(x,y) = xy - x- y + 1
\]
subject to the constraint
\[
px + qy = B\,,
\]
as well as the additional constraints $x \geq 1$ and $y \geq 1$.

We use the method of Lagrange multipliers, which results in the set of equations
\begin{align*}
y - 1 &= p\la &
x - 1 &= q\la &
px + qy &= B\,.
\end{align*}
Substituting the first two equations into the third one yields
\begin{align*}
p(q\la + 1) + q(p\la + 1) &= B \quad\Leftrightarrow\quad
2pq\la = B - p - q \quad\Leftrightarrow\quad
\la = \frac{B-p-q}{2pq}\,.
\end{align*}
Hence we obtain the solution
\begin{align*}
x_0 &= \frac{B+p-q}{2p} &
y_0 &= \frac{B-p+q}{2q}\,,
\end{align*}
and
\[
Q(x_0, y_0) = Q\left(\frac{B+p-q}{2p}, \frac{B-p+q}{2q}\right) = \frac{\left(B-p-q\right)^2}{4pq}\,.
\]

We also have to check that this solution satisfies the additional constraints $x \geq 1$ and $y \geq 1$ and that there are no extrema on the edge, that is with $x=1$ or $y=1$. First we note that
\[
Q(1,y) = y-1-y+1 = 0 \quad\text{and}\quad Q(x, 1) = x-x-1+1 = 0\,,
\]
while $Q(x_0, y_0) \geq 0$, provided $p > 0$ and $q > 0$. We see that we need some additional assumptions on $p$, $q$ and $B$. Since $p$ and $q$ are prices of cotton and wool respectively, it makes sense to assume that $p > 0$ and $q > 0$. The conditions $x \geq 1$ and $y \geq 1$ express that we want to buy at least one pound each of cotton and wool; in order to be able to afford this, we need $B \geq p + q$. So we make the following assumptions
\[
p > 0\,,\quad q > 0 \quad\text{and}\quad B \geq p+q\,.
\]
With these assumptions it is easy to check that $x_0 \geq 1$ and $y_0 \geq 1$ as well as $Q(x_0, y_0) \geq 0$. This means that we have indeed found a maximum satisfying the required constraints.

Therefore most cloth can be produced by buying $\displaystyle \frac{B+p-q}{2p}$ pounds of wool and $\displaystyle \frac{B-p+q}{2q}$ pounds of cotton yielding $\displaystyle \frac{\left(B-p-q\right)^2}{4pq}$ pounds of cloth.
\end{solution}

\begin{question}
\SetQuestionProperties{source = {MA2712 Tests 2015, 2016}}
Find the minimum and maximum values of the following functions subject to the given constraints using the method of Lagrange multipliers. Where are the minimum and maximum attained?
\begin{tasks}(2)
\task
$f(x,y) = xy$; $x^2 + y^2 = 4$
\task
$f(x,y) = x^3 - y^3$; $x^2 + y^2 = 2$
\task
$f(x,y) = 2x^3 - y^3$; $x^2 + y^2 = 5$
\task
$f(x,y) = x e^y$; $x^2 + y^2 = 2$
\end{tasks}
\end{question}

\begin{solution}
\begin{enumerate}
\item
The Lagrange function is $xy-\lambda(x^2+y^2-4)$ and
the equations for Lagrange multipliers are
\begin{align*}
y &= 2x \lambda &
x &= 2y\lambda &
x^2 + y^2 &= 4\,.
\end{align*}
We substitute the second equation into the first to obtain $y = 4y\lambda^2$,
which can be rewritten as
\[
y(4\lambda^2 - 1) = 0.
\]
This leads to the three cases $y=0$, $\lambda = \frac 12$ and $\lambda = - \frac 12$.

\begin{itemize}
\item
If $y=0$, then the second equation gives $x=0$, but this is not compatible with the constraint $x^2 + y^2 = 4$.

\item
If $\lambda = \frac 12$, then the first equation becomes $y=x$ and the constraint gives $2x^2 = 4$ or $x = \pm \sqrt{2}$, leading to the two solutions $(\sqrt{2}, \sqrt{2})$ and $(-\sqrt{2}, -\sqrt{2})$.

\item
If $\lambda = - \frac 12$, then the first equation becomes $y = -x$ and the constraint gives again $x = \pm \sqrt{2}$, leading to the two solutions $(-\sqrt{2}, \sqrt{2})$ and $(\sqrt{2}, -\sqrt{2})$.
\end{itemize}

From
\begin{align*}
f(\sqrt{2}, \sqrt{2}) &= 2 & f(-\sqrt{2}, -\sqrt{2}) &= 2 & f(-\sqrt{2}, \sqrt{2}) &= -2 & f(\sqrt{2}, -\sqrt{2}) = -2\,,
\end{align*}
we see that the minimum value of $f$ is $-2$ and the maximum value of $f$ is $2$, both are attained at two points.

\item
The equations for Lagrange multipliers are
\begin{align*}
3x^2 &= 2x \lambda &
-3y^2 &= 2y\lambda &
x^2 + y^2 &= 2 \,.
\end{align*}
The first two equations can be rewritten as
\begin{align*}
x(2\lambda-3x) &= 0 &
y(2\lambda+3y) &= 0\,.
\end{align*}
which leads to the three cases $x=0$, $y=0$ or both $x,y \neq 0$.

\begin{itemize}
\item
If $x=0$, then the constraint implies $y^2 = 2$ or $y = \pm \sqrt{2}$, giving us the two solutions $(0,\sqrt{2})$ and $(0,-\sqrt{2})$.

\item
If $y=0$, then the constraint implies $x^2 = 2$ or $x = \pm \sqrt{2}$, giving us the two solutions $(\sqrt{2},0)$ and $(-\sqrt{2},0)$.

\item
In the third case we have $x=\frac 23 \lambda$ and $y=-\frac 23 \lambda$. Inserting this into the constraint equation gives $\frac 49 \lambda^2 + \frac 49 \lambda^2 = 2$ or $\lambda^2 = \frac 94$. Hence $\lambda = \pm \frac 32$, leading to the two solutions $(1,-1)$ and $(-1, 1)$.
\end{itemize}

From
\begin{align*}
f(0,\sqrt{2}) &= f(-\sqrt{2}, 0) = -2\sqrt{2} & f(1,-1) &= 2 \\
f(\sqrt{2}, 0) &= f(0,-\sqrt{2}) = 2\sqrt{2} & f(-1,1) &= -2
\end{align*}
we see that the minimum value of $f$ is $-2\sqrt{2}$ and the maximum value of $f$ is $2\sqrt{2}$, both are attained at two points.

\item
The equations for Lagrange multipliers are
\begin{align*}
6x^2 &= 2x \lambda &
-3y^2 &= 2y\lambda &
x^2 + y^2 &= 5 \,.
\end{align*}
The first two equations can be rewritten as
\begin{align*}
2x(\lambda-3x) &= 0 &
y(2\lambda+3y) &= 0\,.
\end{align*}
which leads to the three cases $x=0$, $y=0$ or both $x,y \neq 0$.

\begin{itemize}
\item
If $x=0$, then the constraint implies $y^2 = 5$ or $y = \pm \sqrt{5}$, giving us the two solutions $(0,\sqrt{5})$ and $(0,-\sqrt{5})$.

\item
If $y=0$, then the constraint implies $x^2 = 5$ or $x = \pm \sqrt{5}$, giving us the two solutions $(\sqrt{5},0)$ and $(-\sqrt{5},0)$.

\item
In the third case we have $x=\frac 13 \lambda$ and $y=-\frac 23 \lambda$. Inserting this into the constraint equation gives $\frac 19 \lambda^2 + \frac 49 \lambda^2 = 5$ or $\lambda^2 = 9$. Hence $\lambda = \pm 3$, leading to the two solutions $(1,-2)$ and $(-1, 2)$
\end{itemize}

From
\begin{align*}
f(0,\sqrt{5}) &= -5\sqrt{5} & f(0,-\sqrt{5}) &= 5\sqrt{5} & f(1,-2) &= 10 \\
f(\sqrt{5}, 0) &= 10\sqrt{5} & f(-\sqrt{5},0) &= -10\sqrt{5} & f(-1,2) &= -10
\end{align*}
we see that the minimum value of $f$ is $-10\sqrt{5}$ and the maximum value of $f$ is $10\sqrt{5}$, both are attained at one point.

\item
The equations for Lagrange multipliers are
\begin{align*}
e^y &= 2x \lambda &
xe^y &= 2y\lambda &
x^2 + y^2 &= 2 \,.
\end{align*}
The first equation shows that $x \neq 0$ and $\lambda \neq 0$. We divide the second equation by the first to obtain $x^2 = y$, which we substitute into the constraint equation, leading to the quadratic equation
\[
y^2 + y - 2 = 0\,.
\]
This equation has the two solutions $y=1$ and $y=-2$. If $y=-2$, then we have no real solutions for $x$ in the equation $x^2=y$. Thus the only solutions are $(1,1)$ and $(-1,1)$. From
\begin{align*}
f(1,1) &= e & f(-1,1) &= -e\,,
\end{align*}
we see that the minimum value of $f$ is $-e$ and the maximum value of $f$ is $e$, both are attained at one point.
\end{enumerate}
\end{solution}

% \begin{question}
% Find the points on the surface 
% \[
% x^2-yz=5
% \]
% that are closest to the origin using the method of Lagrange multipliers.

% \hfill
% [Exercise 14.8.36, Anton, 8th ed.]
% \end{question}

% \begin{question}
% Find the points on the curve
% \[
% x^2 + xy + y^2 = 1\,,
% \]
% that are closest to and furthest away from the origin using the method of Lagrange multipliers.

% \hfill
% [Exercise 14.8.8, Thomas, 11th ed.]
% \end{question}

%%% Local Variables:
%%% TeX-master: "problems"
%%% End:
