\begin{question}
Find an example for each of the following domains.
\begin{tasks}(2)
\task
A domain of type 1 and type 2.
\task
A domain of neither type 1 nor type 2.
\task
A domain of type 1, but not of type 2.
\task
A domain of type 2, but not of type 1.
\end{tasks}
\end{question}

\begin{solution}
Examples of domains can be found below.
\begin{center}
\begin{figuretable}{4}
\begin{tikzpicture}[scale=1.3, baseline=(X.base)]
  \def\scale{1.3}
  \node at (0,-1) (X) {};

  \drawaxes{-1}{-1}{1}{1}
  \clip (-1,-1) rectangle (1,1);

  \draw[integration domain] (1,1) |- (-1,-1) |- cycle;
\end{tikzpicture}
&
\begin{tikzpicture}[scale=1.3, baseline=(X.base)]
  \def\scale{1.3}
  \node at (0,-1) (X) {};

  \drawaxes{-1}{-1}{1}{1}
  \clip (-1,-1) rectangle (1,1);

  \draw[integration domain]
    plot[parametric, domain=-1:1, samples=100] function {t, 0.5*t**2+0.5}
    -- plot[parametric, domain=1:-1, samples=100] function {0.5*t**2+0.5, t}
    -- plot[parametric, domain=1:-1, samples=100] function {t, -0.5*t**2-0.5}
    -- plot[parametric, domain=-1:1, samples=100] function {-0.5*t**2-0.5, t};
\end{tikzpicture}
&
\begin{tikzpicture}[scale=1.3, baseline=(X.base)]
  \def\scale{1.3}
  \node at (0,-1) (X) {};

  \drawaxes{-1}{-1}{1}{1}
  \clip (-1,-1) rectangle (1,1);

  \draw[integration domain]
    plot[parametric, domain=-1:1, samples=100] function {t, 0.5*t**2+0.5}
    -- (1,-1)
    -- plot[parametric, domain=1:-1, samples=100] function {t, -0.5*t**2-0.5}
    -- (-1,1);
\end{tikzpicture}
&
\begin{tikzpicture}[scale=1.3, baseline=(X.base)]
  \def\scale{1.3}
  \node at (0,-1) (X) {};

  \drawaxes{-1}{-1}{1}{1}
  \clip (-1,-1) rectangle (1,1);

  \draw[integration domain]
    (-1,1) -- (1,1)
    -- plot[parametric, domain=1:-1, samples=100] function {0.5*t**2+0.5, t}
    -- (1,-1) -- (-1,-1)
    -- plot[parametric, domain=-1:1, samples=100] function {-0.5*t**2-0.5, t};
\end{tikzpicture}
\\
(a) & (b) & (c) & (d)
\end{figuretable}
\end{center}
\end{solution}

\begin{question}
\SetQuestionProperties{source = {Ex. 1--2; MW, III.17.2}}
Sketch each of the following domains and determine whether it is of type 1, type 2, both or neither.
\begin{tasks}(2)
\task
$(x,y)$ such that $0 \leq y \leq 3x$, $0 \leq x \leq 1$.
\task
$(x,y)$ such that $2y^2-1 \leq x \leq y^2$, $|y| \leq 1$.
\task
$(x,y)$ such that $y^2 \leq x \leq y$, $0 \leq y \leq 1$.
\task
$(x,y)$ such that $1 - x^2 \leq 2y$, $x^2+y^2 \leq 1$.
\task
$(x,y)$ such that $x^2 + y^2 \leq 1$.
\task
$(x,y)$ such that $\displaystyle \frac 12 \leq x^2 + y^2 \leq 1$.
\end{tasks}
\end{question}

\begin{solution}
The domains can be seen in the figures below.
% \begin{center}
% \begin{figuretable}{3}
% \begin{tikzpicture}[scale=1.25, baseline=0]
%   \def\scale{1.25}

%   \drawxlabels{1/1}
%   \drawylabels{3/3}
%   \drawaxes{0}{0}{3}{3}

%   \draw[dashed] (0,3) -- (1,3);

%   \draw[integration domain] (0,0) -- (1,0) -- (1,3) -- (0,0);
%   \node at (0.66,1) {$D$};
% \end{tikzpicture}
% &
% \begin{tikzpicture}[scale=1.25, baseline=(X.base)]
%   \def\scale{1.25}
%   \node at (0,-1.5) (X) {};

%   \drawaxes{-1.5}{-1.5}{1.5}{1.5}

%   \node[above left] at (-.5,.5) {$x=2y^2-1$};
%   \node[below right] at (.5,.7) {$x=y^2$};

%   \clip (-1.5,-1.5) rectangle (1.5,1.5);

%   \draw[dashed] (0,1) -- (1,1)
%                 (0,-1) -- (1,-1);

%   \fill[integration domain]
%     plot[parametric, domain=-1:1, samples=100] function {2*t**2-1, -t}
%     --plot[parametric, domain=-1:1, samples=100] function {t**2, t};

%   \draw plot[parametric, domain=-1.5:1.5, samples=100] function {2*t**2-1, t};
%   \draw plot[parametric, domain=-1.5:1.5, samples=100] function {t**2, t};

%   \node at (-0.5,0) {$D$};

%   \drawylabels[nofill]{-1/-1, 1/1}
% \end{tikzpicture}
% &
% \begin{tikzpicture}[scale=2.5, baseline=(X.base)]
%   \def\scale{2.5}
%   \node at (0,-0.25) (X) {};

%   \drawaxes{-0.25}{-0.25}{1.25}{1.25}
%   \drawxlabels{1/1}
%   \drawylabels{1/1}

%   \draw[dashed] (1,0) -- (1,1) -- (0,1);

%   \clip (-0.25,-0.25) rectangle (1.25,1.25);

%   \fill[integration domain] (0,0) -- (1,1) 
%     --plot[domain=1:0, samples=100] function {sqrt(x)};
%   \draw (-0.25,-0.25) -- (1.25,1.25);
%   \draw plot[parametric, domain=-0.25:1.25, samples=100] function {t**2, t};

%   \node[below right] at (0.25,0.25) {$x=y$};
%   \node[above left] at (0.5,0.707) {$x=y^2$};
%   \node at (0.375,0.5) {$D$};
% \end{tikzpicture}
% \\
% (a) & (b) & (c)
% \end{figuretable}
% \end{center}

\begin{center}
\begin{figuretable}{3}
\begin{tikzpicture}[scale=1.25, baseline=0]
  \def\scale{1.25}

  \drawxlabels{1/1}
  \drawylabels{3/3}
  \drawaxes{0}{0}{3}{3}

  \draw[dotted] (0,3) -- (1,3);

  \draw[integration domain] (0,0) -- (1,0) -- (1,3) -- (0,0);
  \node at (0.66,1) {$D$};
\end{tikzpicture}
&
\begin{tikzpicture}[scale=1.25, baseline=(X.base)]
  \def\scale{1.25}
  \node at (0,-1.5) (X) {};

  \drawaxes{-1.5}{-1.5}{1.5}{1.5}

  \node[above left] at (-.5,.5) {$x=2y^2-1$};
  \node[below right] at (.5,.7) {$x=y^2$};

  \clip (-1.5,-1.5) rectangle (1.5,1.5);

  \draw[dotted] (0,1) -- (1,1)
                (0,-1) -- (1,-1);

  \draw[integration domain]
    plot[parametric, domain=-1:1, samples=100] function {2*t**2-1, -t}
    --plot[parametric, domain=-1:1, samples=100] function {t**2, t};

  \draw[dashed]
     plot[parametric, domain=1:1.5, samples=100] function {2*t**2-1, t}
     plot[parametric, domain=-1.5:-1, samples=100] function {2*t**2-1, t};
  \draw[dashed]
    plot[parametric, domain=1:1.5, samples=100] function {t**2, t}
    plot[parametric, domain=-1.5:-1, samples=100] function {t**2, t};

  \node at (-0.5,0) {$D$};

  \drawylabels[nofill]{-1/-1, 1/1}
\end{tikzpicture}
&
\begin{tikzpicture}[scale=2.5, baseline=(X.base)]
  \def\scale{2.5}
  \node at (0,-0.25) (X) {};

  \drawaxes{-0.25}{-0.25}{1.25}{1.25}
  \drawxlabels{1/1}
  \drawylabels{1/1}

  \draw[dotted] (1,0) -- (1,1) -- (0,1);

  \clip (-0.25,-0.25) rectangle (1.25,1.25);

  \draw[integration domain] (0,0) -- (1,1) 
    --plot[domain=1:0, samples=100] function {sqrt(x)};
  \draw[dashed] (-0.25,-0.25) -- (0,0)
                (1,1) -- (1.25,1.25);
  \draw[dashed]
    plot[parametric, domain=-0.25:0, samples=100] function {t**2, t}
    plot[parametric, domain=1:1.25, samples=100] function {t**2, t};

  \node[below right] at (0.5,0.5) {$x=y$};
  \node[above left] at (0.5,0.707) {$x=y^2$};
  \node at (0.375,0.5) {$D$};
\end{tikzpicture}
\\
(a) & (b) & (c)
\end{figuretable}
\end{center}

\begin{enumerate}
\item
The domain is a triangle with vertices $(0,0)$, $(1,0)$ and $(1,3)$. It is a domain of both types.
\item
This domain can be written as
\[
D = \left\{ (x,y) \,:\, -1 \leq y \leq 1,\, 2y^2 - 1 \leq x \leq y^2 \right\}\,.
\]
It is a type 2 domain, but not a type 1 domain.
\item
This domain can be written as
\begin{align*}
D &= \left\{ (x,y) \,:\, 0 \leq y \leq 1\,,\; y^2 \leq x \leq y \right\}
= \left\{ (x,y) \,:\, 0 \leq x \leq 1\,,\; x \leq y \leq \sqrt{x} \right\}\,,
\end{align*}
and hence it is a domain of both types.
\item
The domain can be written as
\[
D = \left\{ (x,y) \,:\, -1 \leq x \leq 1,\, \frac 12\left(1-x^2\right) \leq y \leq \sqrt{1-x^2} \right\}\,.
\]
It is a type 1 domain, but not a type 2 domain.
\item
The domain can be written as
\begin{align*}
D &= \left\{ (x,y) \,:\, -1 \leq y \leq 1\,,\; -\sqrt{1-y^4} \leq x \leq \sqrt{1-y^4} \right\} \\
&= \left\{ (x,y) \,:\, -1 \leq x \leq 1\,,\; -\sqrt{1-x^4} \leq y \leq \sqrt{1-x^4} \right\}\,,
\end{align*}
and it is a domain of both types.
\item
This domain is neither of type 1 nor of type 2.
\end{enumerate}

\begin{center}
\begin{figuretable}{3}
\begin{tikzpicture}[scale=1.875, baseline=(X.base)]
  \def\scale{1.875}
  \node at (0,-1) (X) {};

  \drawaxes{-1}{-1}{1}{1}

  \draw[integration domain] 
    plot[domain=-1:1, samples=100] function {0.5-0.5*x**2}
    --(1,0) arc[start angle=0, end angle=180, radius=1];

  \draw[dashed]
    (-1,0) arc[radius=1, start angle=180, end angle=360];
%  \draw plot[domain=-1:1, samples=100] function {0.5-0.5*x**2};

  \node at (0,0.75) {$D$};

  \drawxlabels[fill]{-1/-1, 1/1}
  \drawylabels[nofill]{1/1}
  \drawylabels[fill]{-1/-1}
\end{tikzpicture}
&
\begin{tikzpicture}[scale=1.875, baseline=(X.base)]
  \def\scale{1.875}
  \node at (0,-1) (X) {};

  \draw[integration domain, samples=100, smooth]
    plot[parametric, domain=0:3.1416/2] 
      function {cos(t), sin(t)}
    --plot[parametric, domain=3.1416/2:3.1416] 
      function {-(abs(cos(t))), sin(t)}
    --plot[parametric, domain=3.1416:3.1416*1.5] 
      function {-abs(cos(t)), -abs(sin(t))}
    --plot[parametric, domain=3.1416*1.5:3.1416*2] 
      function {cos(t), -abs(sin(t))}
    --(1,0) ;

  % Coordinate axes
  \drawaxes{-1}{-1}{1}{1}
  \drawxlabels[nofill]{-1/-1, 1/1}
  \drawylabels[nofill]{-1/-1, 1/1}

  % Label
  \node at (0.4,0.4) {$D$};
\end{tikzpicture}
&
\begin{tikzpicture}[scale=1.875, baseline=(X.base)]
  \def\scale{1.875}
  \node at (-1,-1) (X) {};

  \def\r{0.707106} % sqrt(2)/2

  % Draw both curves at the same time and use even odd rule to plot inside
  \draw[integration domain, samples=100, smooth, even odd rule]
    plot[parametric, domain=0:3.1416/2] 
      function {cos(t), sin(t)}
    --plot[parametric, domain=3.1416/2:3.1416] 
      function {-(abs(cos(t))), sin(t)}
    --plot[parametric, domain=3.1416:3.1416*1.5] 
      function {-abs(cos(t)), -abs(sin(t))}
    --plot[parametric, domain=3.1416*1.5:3.1416*2] 
      function {cos(t), -abs(sin(t))}
    --(1,0)
    plot[parametric, domain=0:3.1416/2] 
      function {\r*cos(t), \r*sin(t)}
    --plot[parametric, domain=3.1416/2:3.1416] 
      function {-\r*(abs(cos(t))), \r*sin(t)}
    --plot[parametric, domain=3.1416:3.1416*1.5] 
      function {-\r*abs(cos(t)), -\r*abs(sin(t))}
    --plot[parametric, domain=3.1416*1.5:3.1416*2] 
      function {\r*cos(t), -\r*abs(sin(t))}
    --(\r,0) ;

  % Coordinate axes
  \drawaxes{-1}{-1}{1}{1}
  \drawxlabels[nofill]{-1/-1, \r/\textstyle{\frac{\sqrt{2}}{2}}, 1/1}
  \drawylabels[nofill]{-1/-1, 1/1}

  % Label
  \node at (0.4,0.75) {$D$};
\end{tikzpicture}
\\
(d) & (e) & (f)
\end{figuretable}
\end{center}
\end{solution}

\begin{question}
\SetQuestionProperties{source = {Ex. 5; Guichard, 15.1}}
Evaluate the following iterated integrals.
\begin{tasks}(2)
\task
$\int_0^1 \int_0^y xy \ud x \ud y$
\task
$\int_1^4 \int_1^{\sqrt{x}} y^2 \ud y \ud x$.
\task
$\int_1^2 \int_{y^2/2}^{\sqrt y} 1 \ud x \ud y$
\task
$\int_1^2 \int_1^x \frac{x^2}{y^2} \ud y \ud x$
\task
$\int_0^1 \int_0^x \frac{y}{e^x} \ud y \ud x$
\task
$\int_0^{\sqrt{\pi}/2} \int_0^{x^2} x \cos y \ud y \ud x$
\end{tasks}
\end{question}

\begin{solution}
\begin{enumerate}
\item
\begin{alignenum}
\int_0^1 \int_0^y xy \ud x \ud y
= \int_0^1 \frac 12 x^2 y \bigg|_{x=0}^y \ud y
= \frac 12 \int_0^1 y^3 \ud y
= \frac 18\,.
\end{alignenum}
\item
\begin{alignenum}  
\int_1^4 \int_1^{\sqrt{x}} y^2 \ud y \ud x
&= \int_1^4 \frac 13 y^3 \bigg|_{y=1}^{\sqrt{x}} \ud x
= \int_1^4 \frac 13 x^{3/2} - \frac 13 \ud x 
= \frac 2{15} x^{5/2} - \frac 13 x \bigg|_{x=1}^4
%= \frac{64}{15} - \frac 4 3 - \frac 2{15} + \frac 13
= \frac{47}{15}\,.
\end{alignenum}
\item
\begin{alignenum}
\int_1^2 \int_{y^2/2}^{\sqrt y} 1 \ud x \ud y
= \int_1^2 \sqrt{y} - \frac 12 y^2 \ud y
= \frac 23 y^{3/2} - \frac 16 y^3 \bigg|_{y=1}^2
= \frac{4}{3}\sqrt{2} - \frac{11}{6}\,.
\end{alignenum}
\item
\begin{alignenum}
\int_1^2 \int_1^x \frac{x^2}{y^2} \ud y \ud x
= \int_1^2 -\frac{x^2}{y} \bigg|_{y=1}^x \ud x
= \int_1^2 -x + x^2 \ud x 
= -\frac 12 x^2 + \frac 13 x^3\bigg|_{x=1}^2
= \frac{5}{6}\,.
\end{alignenum}
\item
\begin{alignenum}
\int_0^1 \int_0^x \frac{y}{e^x} \ud y \ud x
&= \int_0^1 \frac 12 \frac{y^2}{e^x} \bigg|_{y=0}^x \ud x
= \frac 12 \int_0^1 x^2 e^{-x} \ud x
= -\frac 12 x^2 e^{-x}\bigg|_{x=0}^1 + \int_0^1 x e^{-x} \ud x \\
&= -\frac 12 e^{-1} - x e^{-x} \bigg|_{x=0}^1 + \int_0^1 e^{-x} \ud x
= -\frac 32 e^{-x} - e^{-x}|_{x=0}^1 = 1-\frac 52 e^{-x}\,.
\end{alignenum}
\item
\begin{alignenum}
\int_0^{\sqrt{\pi}/2} \int_0^{x^2} x \cos y \ud y \ud x
%= \int_0^{\sqrt{\pi}/2} x \sin y \bigg|_{y=0}^{x^2} \ud x
= \int_0^{\sqrt{\pi}/2} x \sin\left(x^2\right) \ud x
= -\frac 12 \cos \left(x^2\right) \bigg|_{x=0}^{\sqrt{\pi}/2} = \frac 12 - \frac{\sqrt 2}4\,.
\end{alignenum}
\end{enumerate}
\end{solution}

\begin{question}
\SetQuestionProperties{source = {Ex. 9--11; Thomas 12th, 15.2}}
Write an iterated integral $\iint_D f(x,y) \ud x \ud y$ over the described region, both as a type 1 domain and as a type 2 domain.
\begin{center}
\begin{figuretable}{3}
\begin{tikzpicture}[scale=1.875, baseline=(X.base)]
  \def\scale{1.875}
  \node at (0,-0.5) (X) {};

  \drawaxes{-0.5}{-0.5}{1.5}{1.5}
  \clip (-.5,-.5) rectangle (1.5,1.5);

  \draw plot[parametric, domain=-0.5:1.5, samples=100] function {t, t**3};
  \draw (-.5,1) -- (1.5,1);

  \draw[integration domain]
    plot[parametric, domain=0:1, samples=100] function {t, t**3}
    -- (0,1) -- (0,0);

  \node[below right] at (1,1) {$y=8$};
  \node[left] at (1.1, {1.1^3}) {$y=x^3$};

  \node at (0.4,0.6) {$D$};
\end{tikzpicture}
&
\begin{tikzpicture}[scale=1.875, baseline=(X.base)]
  \def\scale{1.875}
  \node at (0,-0.5) (X) {};

  \drawaxes{-0.5}{-0.5}{1.5}{1.5}
  \clip (-.5,-.5) rectangle (1.5,1.5);

  \draw (-.5,-.5) -- (1.5,1.5);
  \draw (1,-.5) -- (1,1.5);

  \draw[integration domain]
    (0,0) -- (1,0) -- (1,1) -- cycle;

  \node[right] at (1,.5) {$x=3$};
  \node[above left] at (.7,.7) {$y=2x$};

  \node at (0.6,0.3) {$D$};
\end{tikzpicture}
&
\begin{tikzpicture}[scale=1.875, baseline=(X.base)]
  \def\scale{1.875}
  \node at (0,-0.5) (X) {};

  \drawaxes{-0.5}{-0.5}{1.5}{1.5}
  \clip (-.5,-.5) rectangle (1.5,1.5);

  \draw plot[parametric, domain=-0.5:1.5, samples=100] function {t, t**2};
  \draw (-.5,-.5) -- (1.5,1.5);

  \draw[integration domain]
    plot[parametric, domain=0:1, samples=100] function {t, t**2}
    -- (0,0);

  \node[left] at (.8,.8) {$y=3x$};
  \node[right] at (0.7, 0.49) {$y=x^2$};

  \node at (0.5,0.375) {$D$};
\end{tikzpicture}
\\
(a) & (b) & (c)
\end{figuretable}
\end{center}
\end{question}

\begin{solution}
\begin{enumerate}
\item
The functions $y=x^3$ and $y=8$ intersect in the point $(2,8)$ and the inverse of $y=x^3$ is $x=\sqrt[3]{y}$. Thus
\[
\iint_D f(x,y) \ud x \ud y
= \int_0^2 \int_{x^3}^8 f(x,y) \ud y \ud x
= \int_0^8 \int_0^{\sqrt[3]{y}} f(x,y) \ud x \ud y\,.
\]
\item
The lines $y=2x$ and $x=3$ intersect in the point $(3,6)$ and the iverse of $y=2x$ is $x=\frac y2$. Thus
\[
\iint_D f(x,y) \ud x \ud y
= \int_0^3 \int_0^{2x} f(x,y) \ud y \ud x
= \int_0^6 \int_{y/2}^3 f(x,y) \ud x \ud y\,.
\]
\item
The functions $y=3x$ and $y=x^2$ intersect in the two points $(0,0)$ and $(3,9)$. The inverse of $y=3x$ is $x=\frac y3$ and the inverse of $y=x^2$ is $x=\sqrt y$. Thus
\[
\iint_D f(x,y) \ud x \ud y
= \int_0^3 \int_{x^2}^{3x} f(x,y) \ud y \ud x
= \int_0^9 \int_{y/3}^{\sqrt y} f(x,y) \ud x \ud y\,.
\]
\end{enumerate}
\end{solution}

\begin{question}
\SetQuestionProperties{source = {Ex. 21, 22; MW, III.17.2 and Ex. 13, 16, 17; Hartman, 13.2}}
Evaluate the following integrals
\begin{tasks}(1)
\task
$\iint_D 3-y \ud x \ud y$; $D$ is the domain between the parabolas $y^2=2x$ and $x^2=2y$.
\task
$\iint_D x^2-y^2 \ud x \ud y$; $D$ is the rectangle with vertices $(-1,-1)$, $(1,-1)$, $(1,1)$, $(-1,1)$.
\task
$\iint_D x^3y - x \ud x \ud y$; $D$ is the half of the circle $x^2 + y^2 \leq 9$ in the first and second quadrants.
\task
$\iint_D x-y \ud x \ud y$; $D$ is the triangle with vertices $(0,0)$, $(1,0)$ and $(2,2)$.
\task
$\iint_D y \cos x \ud x \ud y$; $D$ is the triangle defined by $0 \leq x \leq \pi$ and $0 \leq y \leq x$.
\task
\difficulty{*}
$\iint_D e^y \ud x \ud y$; $D$ is the domain between $y=\ln x$ and $y = \frac{1}{e-1}(x-1)$.
\end{tasks}
\end{question}

\begin{solution}
\begin{center}
\begin{figuretable}{3}
\begin{tikzpicture}[scale=1.25, baseline=(X.base)]
  \def\scale{1.25}
  \node at (0,-.5) (X) {};

  \drawaxes{-.5}{-.5}{2.5}{2.5}
  \clip (-.5,-.5) rectangle (2.5,2.5);

  \draw[dotted] (0,2) -- (2,2) -- (2,0);

  \draw[integration domain] 
    plot[parametric, domain=0:2, samples=100] function {t, sqrt(2*t)}
    -- plot[parametric, domain=2:0, samples=100] function {t, t**2/2};

  \draw[dashed]
    plot[parametric, domain=-0.5:0, samples=100] function {t, t**2/2}
    plot[parametric, domain=2:2.5, samples=100] function {t, t**2/2};
  \draw[dashed]
    plot[parametric, domain=-0.5:0, samples=100] function {t**2/2, t}
    plot[parametric, domain=2:2.5, samples=100] function {t**2/2, t};

  \node[left] at (1.2,{sqrt{2*1.2}}) {$x=y^2/2$};
  \node[right] at (.75,{.7^2/2}) {$x=\sqrt{2y}$};
  \node at (1,1) {$D$};

  \drawxlabels[fill]{2/2}
  \drawylabels[fill]{2/2}
\end{tikzpicture}
&
\begin{tikzpicture}[scale=1.25, baseline=(X.base)]
  \def\scale{1.25}
  \node at (0,-1.5) (X) {};

  \drawaxes{-1.5}{-1.5}{1.5}{1.5}

  \draw[integration domain] (-1,-1) -| (1,1) -| cycle;

  \node[above right] at (1,1) {$(1,1)$};
  \node[below left] at (-1,-1) {$(-1,-1)$};
  
  \node at (0,0) {$D$};
\end{tikzpicture}
&
\begin{tikzpicture}[scale=0.625, baseline=(X.base)]
  \def\scale{0.625}
  \node at (-3,-3) (X) {};

  \draw[integration domain]
    (3,0) arc[radius=3, start angle=0, end angle=180] -- cycle;

  \draw[dashed]
    (3,0) arc[radius=3, start angle=0, end angle=-180];

  % Coordinate axes
  \drawaxes{-3}{-3}{3}{3}
  % \drawxlabels[fill]{-3/-3,3/3}
  % \drawylabels[nofill]{-3/-3,3/3}

  \node[above right] at (80:3) {$y=\sqrt{1-x^2}$};

  % Label
  \node at (1,1.5) {$D$};
\end{tikzpicture}
\\
(a) & (b) & (c)
\end{figuretable}
\end{center}

\begin{enumerate}
\item
The domain can be written as a type 2 domain with lower limit $x=\frac{y^2}2$ and upper limit $x = \sqrt{2y}$. Thus
\begin{align*}
\iint_D 3-y \ud x \ud y
&= \int_0^2 \int_{y^2/2}^{\sqrt{2y}} 3-y \ud x \ud y
= \int_0^2 (3-y)\left(\sqrt{2y} - \frac{y^2}{2} \right) \ud y \\
&= \int_0^2 3\sqrt 2 y^{1/2} - \sqrt 2 y^{3/2} - \frac 32 y^2 + \frac 12 y^3 \ud y \\
&= 2 \sqrt 2 y^{3/2} - \frac 25 \sqrt 2 y^{5/2} - \frac 12 y^3 + \frac 18 y^4 \bigg|_{y=0}^2
= \frac {14}5\,.
\end{align*}
\item
This is a simple iterated integral with fixed limits.
\[
\int_{-1}^1 \int_{-1}^1 x^2 - y^2 \ud x \ud y
= \int_{-1}^1 \frac 13 x^3 - xy^2 \bigg|_{x=-1}^1 \ud y
= \int_{-1}^1 \frac 23 - 2y^2 \ud y = 0\,.
\]
\item
This domain is of type 1 with lower limit $y=0$ and upper limit $y=\sqrt{3-x^2}$. Thus
\begin{align*}
\int_{-3}^3 \int_0^{\sqrt{3-x^2}} x^3 y - x \ud y \ud x
&= \int_{-3}^3 \frac 12 x^3 y^2 - xy \bigg|_{y=0}^{\sqrt{3-x^2}} \ud x
= \int_{-3}^3 \frac 12 x^3 \left(3-x^2 \right) - x \sqrt{3-x^2} \ud x \\
&= \frac 38 x^4 - \frac 1{12}x^6 + \frac 13 \left(3-x^2\right)^{3/2} \bigg|_{x=-3}^3
=0 \,.
\end{align*}
\item
The triangle is a type 2 domain and can be parametrised by
\[
D = \left\{ (x,y) \,:\, 0 \leq y \leq 2\,,\; y \leq x \leq \frac y2+1 \right\}
\]
We then have to calculate the integral
\begin{align*} 
\iint_D x-y \ud x \ud y
&= \int_0^2 \int_{y}^{y/2-2} x-y \ud x \ud y 
= \int_0^2 \frac 12 x^2 - xy \bigg|_{x=y}^{y/2-2} \ud y \\
&= \int_0^1 \frac 12 \left(\frac y2-2\right)^2 - \frac {y^2}{2} - y\left(\frac y2-2\right) + y^2 \ud y \\
&= \int_0^2 \frac{y^2}8 + y + 2 \ud y
= \frac 13 + 2 + 4 = \frac {19}3 \,.
\end{align*}
\item
This integral is best written as an integral over a type 2 domain.
\begin{align*}
\iint_D y \cos x \ud x \ud y
&= \int_0^\pi \int_y^\pi y \cos x \ud x \ud y
= \int_0^\pi y \sin x \bigg|_{x=y}^\pi \ud y
= - \int_0^\pi y \sin y \ud y \\
&= y \cos y \bigg|_{y=0}^\pi + \int_0^\pi \cos y \ud y
= -\pi\,.
\end{align*}
\item
The challenge here is to find the limits of integration. These are given by solutions of
\[
\ln x = \frac {1}{e-1} \left(x-1\right)\,.
\]
The two solutions are given by $x=1$ and $x=e$. The domain is of type 1 and therefore
\begin{align*}
\iint_D e^y \ud x \ud y
&= \int_1^e \int_{\ln x}^{(x-1)/(e-1)} e^y \ud y \ud x
= \int_1^e e^{(x-1)/(e-1)} - x \ud x \\
&= (e-1) e^{(x-1)/(e-1)} - \frac 12 x^2 \bigg|_{x=1}^e
= \frac 32 + \frac 12 e^2\,.
\end{align*}
\end{enumerate}
\begin{center}
\begin{figuretable}{3}
\begin{tikzpicture}[scale=1.875, baseline=(X.base)]
  \def\scale{1.875}
  \node at (0,0) (X) {};

  \drawaxes{0}{0}{2}{2}

  \draw[dotted] (2,0) -- (2,2) -- (0,2);

  \draw[integration domain]
    (0,0) -- (1,0) -- (2,2) -- cycle;

  \node at (1,0.666) {$D$};

  \drawxlabels[fill]{1/1, 2/2}
  \drawylabels[fill]{2/2}
\end{tikzpicture}
&
\begin{tikzpicture}[scale=3.75, baseline=(X.base)]
  \def\scale{3.75}
  \node at (0,0) (X) {};

  \drawaxes{0}{0}{1}{1}

  \draw[dotted] (0,1) -- (1,1);

  \draw[integration domain] 
    (0,0) -- (1,0) -- (1,1) -- cycle;
  
  \node at (0.666,0.333) {$D$};
  
  \drawxlabels{1/\pi};
  \drawylabels{1/\pi};
\end{tikzpicture}
&
\begin{tikzpicture}[scale=1.25, baseline=(X.base)]
  \def\scale{1.25}
  \node at (0,-1.5) (X) {};

  \drawaxes{0}{-1.5}{3}{1.5}

  \drawxlabels{1/1, 2.787/e}
  \drawylabels{1/1}

  \clip (0,-1.5) rectangle (3,1.5);

  \draw[dotted] (2.787,0) -- (2.787,1) -- (0,1);

  \draw[dashed]
    plot[parametric, domain=0.01:1, samples=100] function {t, log(t)}
    plot[parametric, domain=2.787:3, samples=100] function {t, log(t)};

  \draw[dashed] (0,{-1/(2.787-1)}) -- (1,0)
                (2.787,1) -- (3,{2/(2.787-1)});

  \draw[integration domain]
    plot[parametric, domain=1:2.787, samples=100] function {t, log(t)} -- cycle;

  \node at (1.9, 0.55) {$D$};
\end{tikzpicture}
\\
(d) & (e) & (f)
\end{figuretable}
\end{center}
\end{solution}

\begin{question}
Interpret the following iterated integrals as double integrals over a domain $D$. Sketch $D$ and change the order of integration.
\begin{tasks}(2)
\task
$\int_0^3 \int_0^{\sqrt{3}} f(x,y) \ud y \ud x$
\task
$\int_1^3 \int_0^{\ln x} f(x,y) \ud y \ud x$
\task
$\int_0^1 \int_{\arctan y}^{\pi/4} f(x,y) \ud x \ud y$
\task
$\int_{-\sqrt 2}^{\sqrt 2} \int_{y^2-1}^1 f(x,y) \ud x \ud y$
\end{tasks}
\end{question}

\begin{solution}
The domains can be seen below.
\begin{center}
\begin{figuretable}{4}
\begin{tikzpicture}[scale=1.3, baseline=(X.base)]
  \def\scale{1.3}
  \node at (0,0) (X) {};

  \drawaxes{0}{0}{2}{2}

  \draw[integration domain] (0,0) |- (2,1.155) |- cycle;

  \drawxlabels{2/3}
  \drawylabels{1.155/\sqrt{3}}

  \node at (1,.575) {$D$};
\end{tikzpicture}
&
\begin{tikzpicture}[scale=0.8666, baseline=(X.base)]
  \def\scale{0.8666}
  \node at (0,-1) (X) {};

  \drawxlabels{1/1, 3/3}
  \drawylabels{{ln(3)}/\ln 3}
  \drawaxes{0}{-1}{3}{2}

  \clip (0,-1) rectangle (3,2);

  \draw[integration domain]
    plot[parametric, domain=1:3, samples=100] function {t, log(t)}
    |- cycle;
    
  \draw[dashed]
    plot[parametric, domain=0.01:1, samples=100] function {t, log(t)};

  \draw[dotted] (0,{ln(3)}) -- (3,{ln(3)});

  \node at (2.3,0.3) {$D$};
\end{tikzpicture}
&
\begin{tikzpicture}[scale=2.6, baseline=(X.base)]
  \def\scale{2.6}
  \node at (0,0) (X) {};

  \drawylabels{1/1}
  \drawaxes{0}{0}{1}{1}
  \drawxlabels{0.785/\frac{\pi}4}


  \clip (0,0) rectangle (1,1);

  \draw[integration domain]
    plot[parametric, domain=0:1, samples=100] function {atan(t), t}
    |- cycle;

  \draw[dotted] (0,1) -- (0.785,1);

  \node at (.6,.3) {$D$};
\end{tikzpicture}
&
\begin{tikzpicture}[scale=0.8666, baseline=(X.base)]
  \def\scale{0.8666}
  \node at (0,-1.5) (X) {};

  \drawaxes{-1.5}{-1.5}{1.5}{1.5}
  \clip (-1.5,-1.5) rectangle (1.5,1.5);

  \draw[integration domain]
    plot[parametric, domain=-1.414:1.414, samples=100] function {t**2-1, t}
    -- cycle;

  \draw[dotted] (0,-1.414) -- (1,-1.414)
                (0,1.414) -- (1,1.414);

  \drawxlabels[nofill]{-1/-1,1/1}
  \drawylabels{-1.414/-\sqrt{2}, 1.414/\sqrt{2}}

  \node at (0,0) {$D$};
\end{tikzpicture}
\\
(a) & (b) & (c) & (d)
\end{figuretable}
\end{center}

\begin{enumerate}
\item
$\int_0^3 \int_0^{\sqrt{3}} f(x,y) \ud y \ud x
= \int_0^{\sqrt 3} \int_0^{3} f(x,y) \ud x \ud y$
\item
$\int_1^3 \int_0^{\ln x} f(x,y) \ud y \ud x
= \int_0^{\ln 3} \int_{e^y}^3 f(x,y) \ud x \ud y$
\item
$\int_0^1 \int_{\arctan y}^{\pi/4} f(x,y) \ud x \ud y
= \int_0^{\pi/4} \int_0^{\tan x} f(x,y) \ud y \ud x$
\item
$\int_{-\sqrt 2}^{\sqrt 2} \int_{y^2-1}^1 f(x,y) \ud x \ud y
= \int_{-1}^1 \int_{-\sqrt{x+1}}^{\sqrt{x+1}} f(x,y) \ud y \ud x$
\end{enumerate}
\end{solution}

\begin{question}
\SetQuestionProperties{source = {Ex. 17, 19, 20; MW, III.17.2}}
Sketch the region, interchange the order of integration and evaluate the integral.
\begin{tasks}(3)
\task
$\displaystyle \int_0^1 \int_x^1 xy \ud y \ud x$.
\task
$\displaystyle \int_0^1 \int_{1-y}^1 (x+y^2) \ud x \ud y$.
\task
$\displaystyle \int_1^4 \int_1^{\sqrt{x}} \left(x^2 + y^2\right) \ud y \ud x$.
\end{tasks}
\end{question}

\begin{solution} 
The regions are shown in the figures below.

\begin{center}
\begin{tabular}{ccc}
% Triangle (0,0) - (1,0) - (1,1)
\begin{tikzpicture}[scale=3.75, baseline=0]
  \def\scale{3.75}

  \drawaxes{0}{0}{1}{1}
  \drawxlabels{1/1}
  \drawylabels{1/1}

  \draw[black,integration domain] (0,0) -- (1,0) -- (1,1) -- cycle;
  \node[above left] at (0.75, 0.75) {$y=x$};

  \draw[dotted] (0,1) -- (1,1);

  \node at (0.66,.33) {$D$};
\end{tikzpicture}
&
% Triangle (1,0) - (1,1) - (0,1)
\begin{tikzpicture}[scale=3.75, baseline=0]
  \def\scale{3.75}

  \drawaxes{0}{0}{1}{1}
  \drawxlabels{1/1}
  \drawylabels{1/1}

  \draw[integration domain] (1,0) -- (1,1) -- (0,1) -- cycle;
  \node[below left, align=left] at (0.625, 0.375) {$x=1-y$};

  \node at (0.66,.66) {$D$};
\end{tikzpicture}
&
% Area between y=1, y=sqrt(x) and 1<x<4.
\begin{tikzpicture}[scale=0.9375, baseline=(BL)]
  \def\scale{0.9375}
  \coordinate (BL) at (0,-1);
  \coordinate (TR) at (4,3);

  \drawaxes{0}{-1}{4}{3}
  \drawxlabels{1/1, 4/4}
  \drawylabels{1/1, 2/2}

  \draw[dashed]
    plot[domain=0:1, samples=100, smooth] function {sqrt(x)};

  \draw[integration domain, smooth]
    plot[domain=1:4] function {sqrt(x)} |- cycle;
  \draw[dotted] (0,2) -- (4,2)
                (4,0) -- (4,1)
                (1,0) |- (0,1);

  \node[above] at (4,2) {$y=\sqrt{x}$};
  \node at (3,1.4) {$D$};
\end{tikzpicture}
\\
(a) & (b) & (c)
\end{tabular}
\end{center}

\begin{enumerate}
\item
We can write the domain as
\begin{align*}
D &= \left\{ (x,y) \,:\, 0 \leq x \leq 1\,,\; x \leq y \leq 1 \right\} \\
&= \left\{ (x,y) \,:\, 0 \leq y \leq 1\,,\; 0 \leq x \leq y \right\}\,.
\end{align*}
The integral then becomes
\[
\int_0^1 \int_x^1 xy \ud y \ud x = \int_0^1 \int_0^y xy \ud x \ud y
\]
We evaluate the latter integral
\begin{align*}
\int_0^1 \int_0^y xy \ud x \ud y
&= \int_0^1 \frac 12 x^2 y\bigg|_{x=0}^y \ud y
= \int_0^1 \frac 12 y^3 \ud y = \frac 18\,.
\end{align*}

\item
We can write the domain as
\begin{align*}
D &= \left\{ (x,y) \,:\, 0 \leq y \leq 1\,,\; 1-y \leq x \leq 1 \right\} \\
&= \left\{ (x,y) \,:\, 0 \leq x \leq 1\,,\; 1-x \leq y \leq 1 \right\}\,.
\end{align*}
The integral then becomes
\[
\int_0^1 \int_{1-y}^1 \left(x+y^2\right) \ud x \ud y = \int_0^1 \int_{1-x}^1 \left(x+y^2\right) \ud y \ud x \,.
\]
We evaluate the latter integral
\begin{align*}
\int_0^1 \int_{1-x}^1 \left(x+y^2\right) \ud y \ud x
&= \int_0^1 xy + \frac 13 y^3 \bigg|_{y=1-x}^1 \ud x
= \int_0^1 x^2 + \frac 13 - \frac 13 \left(1 - x\right)^3 \ud x \\
&= \frac 13 x^3 + \frac 13 x + \frac 1{12} \left(1-x\right)^4 \bigg|_{x=0}^1
= \frac 7{12}\,.
\end{align*}

\item
We can write the domain as
\begin{align*}
D &= \left\{ (x,y) \,:\, 1 \leq y \leq 4\,,\; 1 \leq y \leq \sqrt x \right\} \\
&= \left\{ (x,y) \,:\, 1 \leq y \leq 2\,,\; y^2 \leq x \leq 4 \right\}\,.
\end{align*}
The integral then becomes
\[
\int_1^4 \int_1^{\sqrt{x}} \left(x^2 + y^2\right) \ud y \ud x
= \int_1^2 \int_{y^2}^4 \left(x^2 + y^2 \right) \ud x \ud y\,.
\]
We evaluate the latter integral
\begin{align*}
\int_1^2 \int_{y^2}^4 \left(x^2 + y^2 \right) \ud x \ud y
&= \int_1^2 \frac 13 x^3 + y^2 x \bigg|_{x=y^2}^4 \ud y
= \int_1^2 \frac{64}3 - \frac 13 y^6 + 4y^2 - y^4 \ud y \\
&= \frac{64}{3} y - \frac 1{21}y^7 + \frac 43 y^3 - \frac 15 y^5 \bigg|_{y=1}^2
= \frac{1934}{105}\,.
\end{align*}
\end{enumerate}
\end{solution}

\begin{question}
\SetQuestionProperties{source = {MA2815 Exams, 2015}}
Evaluate the following integrals.
\begin{tasks}(2)
\task
$\int_{-1}^1 \int_1^2 \frac{x \tan^2 y}{1 + \ln y} \ud y \ud x$
\task
$\int_0^1 \int_{2x}^2 e^{-y^2} \ud y \ud x$
\task
$\int_0^2 \int_{y/2}^1 \cos\left(x^2\right) \ud x\ud y$
\task
$\int_0^1 \int_{\sqrt[3]{y}}^1 \sqrt{x^4+1} \ud x \ud y$
\task
$\int_0^2 \int_{y/2}^1 y \cos (x^3 - 1) \ud x \ud y$
\task
$\int_0^{\pi/2} \int_x^{\pi/2} \frac{\sin y}{y} \ud y \ud x$
\end{tasks}
\end{question}

\begin{solution}
\begin{center}
\begin{figuretable}{4}
\begin{tikzpicture}[scale=1.3, baseline=(X.base)]
  \def\scale{1.3}
  \node at (0,0) (X) {};

  \drawxlabels{1/1}
  \drawylabels{2/2}

  \drawaxes{0}{0}{2}{2}

  \draw[integration domain] (0,0) -- (1,2) -| cycle;

  \draw[dotted] (1,0) -- (1,2);

  \node at (0.333, 1.333) {$D$};
\end{tikzpicture}
&
\begin{tikzpicture}[scale=1.3, baseline=(X.base)]
  \def\scale{1.3}
  \node at (0,0) (X) {};

  \drawxlabels{1/1}
  \drawylabels{2/2}

  \drawaxes{0}{0}{2}{2}

  \draw[integration domain] (0,0) -- (1,2) |- cycle;

  \draw[dotted] (0,2) -- (1,2);

  \node at (0.666, 0.666) {$D$};
\end{tikzpicture}
&
\begin{tikzpicture}[scale=2.6, baseline=(X.base)]
  \def\scale{2.6}
  \node at (0,0) (X) {};

  \drawylabels{1/1}
  \drawxlabels{1/1}
  \drawaxes{0}{0}{1}{1}

  \clip (0,0) rectangle (1,1);

  \draw[integration domain]
    plot[parametric, domain=0:1, samples=100] function {t, t**3}
    |- cycle;

  \draw[dotted] (0,1) -- (1,1);

  \node at (.8,.3) {$D$};
\end{tikzpicture}
&
\begin{tikzpicture}[scale=2.6, baseline=(X.base)]
  \def\scale{2.6}
  \node at (0,0) (X) {};

  \node[left] at (0,.95) {$\textstyle\frac{\pi}2$};
  \node[below] at (1,0) {$\textstyle\frac{\pi}2$};
  \drawaxes{0}{0}{1}{1}
  \clip (0,0) rectangle (1,1);

  \draw[integration domain]
    (0,0) -- (1,1) -- (0,1) -- cycle;

  \draw[dotted] (1,0) -- (1,1);

  \node at (0.333,0.666) {$D$};
\end{tikzpicture}
\\
(b) & (c) \& (e) & (d) & (f)
\end{figuretable}
\end{center}
\begin{enumerate}
\item
\begin{alignenum}
\int_{-1}^1 \int_1^2 \frac{x \tan^2 y}{1 + \ln y} \ud y \ud x
= \int_1^2 \int_{-1}^1 \frac{x \tan^2 y}{1 + \ln y} \ud x \ud y
= \int_1^2 \frac 12 x^2 \cdot \frac{\tan^2 y}{1 + \ln y} \bigg|_{x=-1}^1 \ud y
= 0\,.
\end{alignenum}
\item
\begin{alignenum}
\int_0^1 \int_{2x}^2 e^{-y^2} \ud y \ud x
&= \int_0^2 \int_0^{y/2} e^{-y^2} \ud x \ud y
= \int_0^2 e^{-y^2} x \bigg|_{x=0}^{y/2} \ud y
= \int_0^2 e^{-y^2} \frac y2 \ud y \\
&= -\frac 14 e^{-y^2} \bigg|_{y=0}^2
= \frac 14 \left( 1 - e^{-4}\right)\,.
\end{alignenum}
% The region of integration can be expressed as
% \[
% \{ (x,y) \,:\, 0 \leq x \leq 1,\, 2x \leq y \leq 2 \}
% = \left\{ (x,y) \,:\, 0 \leq y \leq 2,\, 0 \leq x \leq \frac y2 \right\}\,.
% \]
% Interchanging the order of integration leads to
% \[
% I = \int_0^2 \int_0^{y/2} e^{-y^2} \ud x \ud y\,.
% \]
% The first integration is simple
% \[
% I = \int_0^2 e^{-y^2} x \bigg|_{x=0}^{y/2} \ud x = \int_0^2 e^{-y^2} \frac y2 \ud x
% \]
% Now we are also able to perform the $y$-integration
% \[
% I = -\frac 14 \int_0^2 -2y e^{-y^2} \ud y
% = -\frac 14 e^{-y^2} \bigg|_{y=0}^2 = \frac 14 \left( 1 - e^{-4}\right)\,.
% \]
\item
\begin{alignenum}
\int_0^2 \int_{y/2}^1 \cos\left(x^2\right) \ud x\ud y
&= \int_0^1 \int_0^{2x} \cos\left(x^2\right) \ud y \ud x
= \int_0^1 \cos\left( x^2 \right) y \bigg|_{y=0}^{2x} \ud x \\
&= \int_0^1 2x \cos\left(x^2\right) \ud x
= \sin\left(x^2\right) \bigg|_{x=0}^1 = \sin 1\,.
\end{alignenum}
% The region of integration can be expressed as
% \[
% \left\{ (x,y) \,:\, 0 \leq y \leq 2,\, \frac y2 \leq x \leq 1 \right\}
% = \{ (x,y) \,:\, 0 \leq x \leq 1 ,\, 0 \leq y \leq 2x \}\,.
% \]
% Interchanging the order of integration leads to
% \[
% I = \int_0^1 \int_0^{2x} \cos\left(x^2\right) \ud y \ud x\,.
% \]
% The first integration is simple
% \[
% I = \int_0^1 \cos\left( x^2 \right) y \bigg|_{y=0}^{2x} \ud x
% = \int_0^1 2x \cos\left(x^2\right) \ud x\,.
% \]
% Now we are also able to perform the $x$-integration
% \[
% I = \sin\left(x^2\right) \bigg|_{x=0}^1 = \sin 1\,.
% \]
\item
\begin{alignenum}
\int_0^1 \int_{\sqrt[3]{y}}^1 \sqrt{x^4+1} \ud x \ud y
&= \int_0^1 \int_0^{x^3} \sqrt{x^4+1} \ud y \ud x
= \int_0^1 y \sqrt{x^4 + 1} \,\bigg|_{y=0}^{x^3} \ud x \\
&= \int_0^1 x^3 \sqrt{x^4 + 1} \ud x
= \frac 16 \left(x^4 + 1\right)^{3/2} \bigg|_{x=0}^1
= \frac 16 \left(2\sqrt{2} - 1\right)\,.
\end{alignenum}
\item
\begin{alignenum}
\int_0^2 \int_{y/2}^1 y \cos \left(x^3 - 1 \right) \ud x \ud y
&= \int_0^1 \int_0^{2x} y \cos \left(x^3 - 1 \right) \ud y \ud x
= \int_0^1 \frac 12 y^2 \cos \left( x^3 - 1 \right) \bigg|_{y=0}^{2x} \ud x \\
&= \int_0^1 2x^2 \cos \left(x^3 - 1 \right) \ud x
= \frac 23 \sin \left(x^3 - 1 \right) \bigg|_{x=0}^1
= \frac 23 \sin 1\,.
\end{alignenum}
\item
\begin{alignenum}
\int_0^{\pi/2} \int_x^{\pi/2} \frac{\sin y}{y} \ud y \ud x
&= \int_0^{\pi/2} \int_0^{y} \frac{\sin y}{y} \ud x \ud y
= \int_0^{\pi/2} x \cdot \frac{\sin y}{y} \bigg|_{x=0}^y \ud y \\
&= \int_0^{\pi/2} \sin y \ud y = 1\,.
\end{alignenum}
\end{enumerate}
\end{solution}

\begin{question}
\SetQuestionProperties{difficulty = {*}}
Evaluate the iterated integral
\[
\int_0^2 \int_0^{2\sqrt{4-x^2}} \left( 4-x^2 - \frac 14 y^2\right) \ud y \ud x\,.
\]
\begin{hint*}
After performing the integration with respect to $y$, use a change of coordinates to transform the square roots from $\sqrt{4-x^2}$ to $\sqrt{1-u^2}$. Then use the trigonometric subsitution. To evaluate the resulting integral you will need to apply the half-angle formula twice.
\end{hint*}
\end{question}

\begin{solution}
We do the calculations step by step.
\begin{align*}
\int_0^2 \int_0^{2\sqrt{4-x^2}} \left( 4-x^2 - \frac 14 y^2\right) \ud y \ud x 
&= \int_0^2 8 \sqrt{4-x^2} - 2x^2 \sqrt{4-x^2} - \frac 8{12} \left(\sqrt{4-x^2}\right)^3 \ud x \\
&= \int_0^2 \frac 43 \left( \sqrt{4-x^2} \right)^3 \ud x
= \int_0^2 \frac {32}3 \left( \sqrt{1-\frac{x^2}4} \right)^3 \ud x\,.
\end{align*}
Now make the substitution $\frac x2 = u$ with $\ud x = 2 \ud u$. We obtain
\[
\int_0^2 \frac {32}3 \left( \sqrt{1-\frac{x^2}4} \right)^3 \ud x = 2 \int_0^1 \frac {32}3 \left( \sqrt{1-u^2} \right)^3 \ud u\,.
\]
Next we use the trigonometric substitution $u = \sin t$ with $\ud u = \cos t \ud t$. This leads to
\[
2 \int_0^1 \frac {32}3 \left( \sqrt{1-u^2} \right)^3 \ud u =
2 \int_0^{\pi/2} \frac {32}3 \left( \sqrt{1-\sin^2 t} \right)^3 \cos t \ud u =
\frac{64}3 \int_0^{\pi/2}  \cos^4 t \ud u\,.
\]
To evaluate this integral we use the half-angle formula twice to get
\begin{align*}
\cos^4 t &= \left( \frac{1 + \cos 2t}2 \right)^2 = \frac 14 + \frac 12 \cos 2t + \frac 14 \cos^2 2t \\
&= \frac 14 + \frac 12 \cos 2t + \frac 14 \frac{1 + \cos 4t}2 = \frac 38 + \frac 12 \cos 2t + \frac 18 \cos 4t\,.
\end{align*}
This is now easy to integrate
\[
\frac{64}3 \int_0^{\pi/2} \frac 38 + \frac 12 \cos 2t + \frac 18 \cos 4t \ud u =
\frac{64}3 \left. \left( \frac 38 t + \frac 14 \sin 2t + \frac 1{32} \sin 4t \right) \right|_{t=0}^{\pi/2} = 4 \pi\,.
\]
\end{solution}

\begin{question}
\SetQuestionProperties{difficulty = {*}}
\SetQuestionProperties{source = {Ex. 28; MW, III.17.2}}
Prove
$\int_0^x \int_0^t f(u) \ud u \ud t = \int_0^x (x-u) f(u) \ud u\,.$
\end{question}

\begin{solution}
\begin{adjustbox}{valign=T,raise=\strutheight,minipage={\linewidth}}
  \begin{wrapfigure}{r}{0.35\textwidth}
    \centering
\begin{tikzpicture}[scale=2.6, baseline=(X.base)]
  \def\scale{2.6}
  \node at (0,0) (X) {};

  \drawxaxis[$t$]{0}{1}
  \drawyaxis[$u$]{0}{1}
  \drawxlabels{1/x}
  \drawylabels{1/x}
  \clip (0,0) rectangle (1,1);

  \draw[integration domain]
    (0,0) -- (1,1) -- (0,1) -- cycle;

  \draw[dotted] (1,0) -- (1,1);

  \node at (0.333,0.666) {$D$};
\end{tikzpicture}
\end{wrapfigure}
\strut{}
Treating $x$ as a constant, we change the order of integration
\begin{align*}
\int_0^x \int_0^t f(u) \ud u \ud t 
&= \int_0^x \int_u^x f(u) \ud t \ud u \\
&= \int_0^x f(u) \left( \int_0^x 1 \ud t \right) \ud u \\
&= \int_0^x (x-u) f(u) \ud u\,.
\end{align*}
\end{adjustbox}
\end{solution}


%%% Local Variables:
%%% mode: latex
%%% TeX-master: "problems"
%%% End:
