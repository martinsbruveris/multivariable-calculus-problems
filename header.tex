%--------------------------------------------------------------------
%--- Notes to myself
%--------------------------------------------------------------------

% TODO: Tautasdziesma
% TODO: Figure out how vertical space works in title page
%       Clean up code for title page
% TODO: Add section numbers to pdf bookmarks
% TODO: Change document class to book to have two parts
%       Part 1 - problems
%       Part 2 - solutions
%       Both in one document
% TODO: Work on page geometry

% Need gnuplot to compile this
% TODO: Add id to plots, so we can compile without gnuplot,
%       see TikZ manual p.329.
% TODO: Change figuretable to use tabu
%       Avoid globally resetting \tabcolsep
% TODO: Tables for local extrema look bad
% 
% TODO: Sect. 10, choose better functions to integrate over
%       rectangles
% TODO: Check for consistent usage of alignenum
% 
%--------------------------------------------------------------------
%--- Loading packages
%--------------------------------------------------------------------

\PassOptionsToPackage{dvipsnames}{xcolor} % Is loaded by TikZ                                

\usepackage{amsmath,amsthm,textcomp}

\usepackage[charter,expert]{mathdesign}   % Bitstream Charter font, 
                                          % as recommended by Aleksey
\usepackage[scaled=.96,osf]{XCharter}     % Matches the size used in math
\usepackage{microtype}                    % Typographical improvements
\usepackage{bm}

\usepackage[T1]{fontenc}  % These three packages load Times fonts
% \usepackage{mathptmx}
% \usepackage{tgtermes} 

% \usepackage{newtxtext}  % Other fonts to consider
% \usepackage{newtxmath}

\usepackage[utf8]{inputenc}
\usepackage{graphicx}

\usepackage{enumitem}                     % Customize lists
% \usepackage{enumerate}                  
\usepackage{geometry}                     % Customize page geometry
% \usepackage{showframe}                    % To show page geometry
\usepackage{fancyhdr}                     % Custom headers and footers
\usepackage{lastpage}                     % To show "Page 1 / 20"

\usepackage{tikz}                         % For drawing figures

\usepackage{exsheets}                     % For problem/solution environments
\usepackage{wrapfig}                      % To let text wrap around a figure
\usepackage{adjustbox}                    % For wrapfigure inside enumerate
\usepackage{ifthen}                       % iften for defining commands
\usepackage{siunitx}                      % To typeset units
\usepackage{tabu}                         % tabu environment for tables
\usepackage{tasks}                        % tasks environment for horizontal lists
\usepackage{nth}
\usepackage[savepos]{zref}                % For align in enumerate workaround
\usepackage{etoolbox}                     % For parameter toggles
\usepackage{xcolor}                       % For red title

\usepackage[style=numeric, minnames=3,
            doi=false, url=false, isbn=false,
            firstinits=true,
            sortcites=true,
            backend=biber]{biblatex}      % For bibliographies

%--------------------------------------------------------------------
%--- Pdf options
%--------------------------------------------------------------------

\usepackage[
  pdftitle={\docclass{} -- \doctitle}, 
  pdfauthor={\docauthor},
  colorlinks=true,
  linkcolor=blue,
  urlcolor=blue,
  citecolor=blue,
  bookmarks=true,
  bookmarksopenlevel=2]{hyperref}

\usepackage[inline]{asymptote}            % For asymptote graphics

%--------------------------------------------------------------------
%--- Global page geometry and layout
%--------------------------------------------------------------------

\geometry{
%  total={210mm,297mm},
  left=25mm,
  right=25mm,
  bindingoffset=0mm, 
  top=20mm,
  bottom=20mm
}

\fancyhf{}
\renewcommand{\headrulewidth}{0.5pt}
\fancyhead[L]{\textsc{\docclass}}
\fancyhead[C]{\textsc{\doctitle}}
\fancyhead[R]{\textsc{\docdate}}
\renewcommand{\footrulewidth}{0.5pt}
\fancyfoot[L]{\textsc{\docauthor}}
\fancyfoot[C]{}
\fancyfoot[R]{\textsc{Page \thepage\ /\ \pageref*{LastPage}}}

\setlength{\headheight}{14pt} % Was necessary, warning otherwise

\fancypagestyle{firstpage}{
  \fancyhf{}
  \renewcommand{\footrulewidth}{0pt}
  \fancyfoot[L]{}
  \fancyfoot[C]{}
  \fancyfoot[R]{}
  \renewcommand{\headrulewidth}{0mm}
}

\pagestyle{fancy}

\author{\docauthor}
\date{\docdate}
\title{\docclass{} -- \doctitle}

% Custom Title
\newcommand{\linia}{\rule{\linewidth}{0.5pt}}

\makeatletter
\renewcommand{\maketitle}{
\begin{center}
\vspace{2ex}
{\huge \textsc{\@title}}
\vspace{1ex}
\\
\linia\\
\textsc{\@author} \hfill \textsc{\@date}
\vspace{4ex}
\end{center}
}
\makeatother

%--------------------------------------------------------------------
%--- Microtype settings
%--- Source: microtype documentation
%--------------------------------------------------------------------

\microtypesetup{expansion=alltext, step=1}
\microtypesetup{kerning=true}

\DeclareMicrotypeSet*[protrusion]
  { doc }
  { encoding = {*, TS1, OMS},
    family   = {rm*, tt*},
    size     = {footnotesize, small, normalsize} }
\SetProtrusion
  { encoding = OMS,
    family   = mdbch }
  { "68 = {400, },  % \langle
    "69 = { ,400} } % \rangle
\DeclareMicrotypeSet*[kerning]
  { doc }
  { encoding = T1,
    family   = blg, % typewriter font and ...
    font     = * }  % French sample in section \ref{sub:kerning}
\SetExtraKerning
  { encoding = T1,
    family   = blg }
  { _ = {100,100} } % underscores shouldn't touch

%--------------------------------------------------------------------
%--- Global parameters and commands
%--------------------------------------------------------------------

\linespread{1.2}

\everymath{\displaystyle}

% Table to typeset tikz figures; parameter is number of columns
\newenvironment{figuretable}[2][1em]
  {
    \begin{tabular}{*{#2}{c@{\hspace{#1}}}}
  }
  {
    \end{tabular}
  }

% A better solution should be found for this...
\tabcolsep=0pt

% Used to adjust vertical height for wrapfigure inside enumerate
\newlength{\strutheight}
\settoheight{\strutheight}{\strut}

\renewcommand{\labelenumi}{(\alph{enumi})} % Use letters for enumerate

\settasks{counter-format=(tsk[a])} % Use letters for tasks lists
\settasks{label-width={1.5em}}

\let\six=\si % \si is defined by package siunitx, but we use it for \sigma

% Pagebreak after each problem section
\newboolean{sectionnewpage}
\setboolean{sectionnewpage}{false}

% For printing difficulty level
\newcommand{\difficulty}[1]{(#1)}

% Bibliography
\addbibresource{bibliography.bib}

% Align in enumerate environment
% Source: tex.stackexchange.com/questions/9394
\newenvironment{alignenum}{%
  $\begin{aligned}[t]
}{%
  \end{aligned}$
}
% This one has centered formulas
% \newenvironment{alignenum}{%
%   \hfill$\begin{aligned}[t]
% }{%
%   \end{aligned}$\hfill\null
% }

\newcounter{dummy} % necessary for correct hyperlinks (to index, bib, etc.)

%--------------------------------------------------------------------
%--- Question/Solution environments
%--------------------------------------------------------------------

% From tex.stackexchange.com/questions/249455 
% to enable displaying question source
\ExplSyntaxOn
\RenewDocumentCommand{\QuestionNumber}{sm}
 {
  \IfBooleanTF{#1}
   { \exsheets_question_number:o {#2} }
   { \exsheets_question_number:n {#2} }
 }
\cs_generate_variant:Nn \exsheets_question_number:n { o }

\RenewDocumentCommand{\IfQuestionPropertyTF}{smmmm}
 {
  \IfBooleanTF{#1}
   { \exsheets_if_question_property:noTF { #2 } { #3 } { #4 } { #5 } }
   { \exsheets_if_question_property:nnTF { #2 } { #3 } { #4 } { #5 } }
 }
\cs_generate_variant:Nn \exsheets_if_question_property:nnTF { no }
\ExplSyntaxOff

% To display question source
\DeclareQuestionProperty{source}
\DeclareQuestionProperty{difficulty}

\SetupExSheets{
  question/post-body-hook={%
    \IfQuestionPropertyTF*%
      {source}%
      {\CurrentQuestionID}%
      {\strut\hfill \mbox{[\GetQuestionProperty{source}{\CurrentQuestionID}]}}%
      {}%
    } ,
  question/pre-body-hook={%
    \IfQuestionPropertyTF*%
      {difficulty}%
      {\CurrentQuestionID}%
      {{\bfseries\difficulty{\GetQuestionProperty{difficulty}{\CurrentQuestionID}}}}%
      {}%
  } 
}

\DeclareInstance{exsheets-heading}{block-nonr}{default}{
  title-post-code = {\bfseries .} ,
  attach = {
    main[l,vc]title[l,vc](0pt,0pt) ;
    main[r,vc]points[l,vc](\marginparsep,0pt)
  }
}

\DeclareInstance{exsheets-heading}{runin-nonr}{default}{
  runin = true ,
  title-post-code = {\bfseries .\space} ,
  attach = {
    main[r,vc]points[l,vc](\marginparsep,0pt)
  } ,
  join = {
    main[r,vc]title[r,vc](0pt,0pt)
  }
}

\RenewQuSolPair
  {question}[headings=margin-nr]
  {solution}[headings=runin-nonr]

%--------------------------------------------------------------------
%--- Some theorem-like environments
%--------------------------------------------------------------------

\newtheoremstyle{theoremnote}%
  {\parskip}%   Space above
  {0pt}%        Space below
  {}%           Body font
  {\parindent}% Indent amount
  {\itshape}%   Theorem head font
  {:}%          Punctuation after theorem head
  {.5em}%       Space after theorem head
  {\thmnumber{#2 }\thmname{#1}\thmnote{. #3}}%  Theorem head spec

\theoremstyle{theoremnote}
\newtheorem*{note*}{Note}

\newtheoremstyle{theoremhint}%
  {\parskip}%   Space above
  {0pt}%        Space below
  {}%           Body font
  {}% Indent amount
  {\itshape}%   Theorem head font
  {:}%          Punctuation after theorem head
  {.5em}%       Space after theorem head
  {\thmnumber{#2 }\thmname{#1}\thmnote{. #3}}%  Theorem head spec

\theoremstyle{theoremhint}
\newtheorem*{hint*}{Hint}


%--------------------------------------------------------------------
%--- Tikz setup and parameters
%--------------------------------------------------------------------

\usetikzlibrary{intersections}
\usetikzlibrary{calc}
\usetikzlibrary{math}
\usetikzlibrary{arrows.meta}
\usetikzlibrary{patterns}
\usetikzlibrary{external}        % To create standalone pdf files
 

% \tikzset{external/disable dependency files=false}
% \tikzexternalize % Activate externalization
\tikzsetexternalprefix{tmp_tikz/}
\tikzset{external/optimize command away=\asyinclude}
\tikzset{external/optimize command away=\asy} % Unclear if it will work

% Global parameters for figures
\tikzset{
  integration domain/.style = {fill=blue!30}
}

\tikzset{
  integration domain2/.style = {fill=red!30}
}

\tikzset{
  coordinate grid/.style = {color=gray!40, thin}
}

\tikzset{
  every node/.style = {font = \footnotesize}
}

% Arrowhead, mostly for coordinate axes
\tikzset{
    >={Stealth[length=2mm]}
}

% Arrowtips for vector fields; should provided \scale is defined locally
\tikzset{
  vector field/.style={
    ->, blue,
    >={Straight Barb[length={1.5pt}, width={1.5pt}]}
  },
  vector field/.default=\scale
}

\newcommand{\drawaxes}[4]{
  \draw[->,>={Stealth[length={4pt}]}]
    ({#1-0.1*(#3-#1)},0) -- ({#3+0.1*(#3-#1)},0) node[below] {$x$};
  \draw[->,>={Stealth[length={4pt}]}] 
    (0,{#2-0.1*(#4-#2)}) -- (0,{#4+0.1*(#4-#2)}) node[left] {$y$};
}

\newcommand{\drawaxeslabelled}[6]{
  \draw[->,>={Stealth[length={4pt}]}]
    ({#1-0.1*(#3-#1)},0) -- ({#3+0.1*(#3-#1)},0) node[below] {#5};
  \draw[->,>={Stealth[length={4pt}]}] 
    (0,{#2-0.1*(#4-#2)}) -- (0,{#4+0.1*(#4-#2)}) node[left] {#6};
}

\newcommand{\drawxaxis}[3][$x$]{
  \draw[->,>={Stealth[length={4pt}]}]
    ({#2-0.1*(#3-#2)},0) -- ({#3+0.1*(#3-#2)},0) node[below] {#1};
}

\newcommand{\drawyaxis}[3][$y$]{
  \draw[->,>={Stealth[length={4pt}]}] 
    (0,{#2-0.1*(#3-#2)}) -- (0,{#3+0.1*(#3-#2)}) node[left] {#1};
}

\newcommand{\drawxlabels}[2][fill]{
  \foreach \x/\xtext in {#2} {
    \ifthenelse{\equal{#1}{fill}}
    {
      \draw[shift={(\x,0)}] (0pt,{2.5pt/\scale}) -- (0pt,{-2.5pt/\scale}) 
        node[below,fill=white] {$\xtext$};
    }
    {
      \draw[shift={(\x,0)}] (0pt,{2.5pt/\scale}) -- (0pt,{-2.5pt/\scale}) 
        node[below] {$\xtext$};
    }
  }
}

\newcommand{\drawylabels}[2][fill]{
  \foreach \y/\ytext in {#2} {
    \ifthenelse{\equal{#1}{fill}}
    {
      \draw[shift={(0,\y)}] ({2.5pt/\scale},0pt) -- ({-2.5pt/\scale},0pt) 
        node[left,fill=white] {$\ytext$};
    }
    {
      \draw[shift={(0,\y)}] ({2.5pt/\scale},0pt) -- ({-2.5pt/\scale},0pt) 
        node[left] {$\ytext$};
    }
  }
}

\newcommand{\drawpoint}[1]{
 \draw[fill=white] #1 circle [radius=2.5pt/\scale];
}

% Scale of integration regions
% (extent of x-axis) * (scale) = 3.75 = 15/4

% Margin for coordinate axes: 10% in each direction

% Baseline for figure to be set at bottom edge of coordinate grid

%--------------------------------------------------------------------
%--- Global settings for Asymptote graphics
%--------------------------------------------------------------------

\def\asydir{tmp_asy}

\begin{asydef}
import palette;
import three;
import graph3;
import grid3;
import solids;

texpreamble("\usepackage[charter,expert]{mathdesign}");

triple XZplane(pair z) {return (z.x,0,z.y);}

defaultpen(fontsize(8pt));

// Materials for surface patches
// 0 .. (blue)                emphasized region
// 1 .. (grey)                nonemphasized region
// 2 .. (not drawn)
// 3 .. (light grey, trans)   nonemphasized, transparent
// 4 .. (light red, trans)    auxilary surface
material[] surface_pens =
	new material[] {lightblue+opacity(1.),
									material(diffusepen=lightgray+opacity(1.), 
													 emissivepen=gray(0.3),
													 specularpen=gray(0.2)),
									lightgrey+opacity(0.),
									lightgrey+opacity(0.8),
									red+opacity(0.2)};
material aux_surface_pen = surface_pens[4];

int emph_ind = 0;
int grey_ind = 1;
int trns_ind = 2;
int grtr_ind = 3;
int auxs_ind = 4;

// Mesh lines on surface
pen mesh_pen = black;
// Color of grid in xy-plane
pen xygrid_pen = grey;
// Color of vertical lines
pen support_lines_pen = grey;
// Mesh lines on auxilary surface
pen aux_mesh_pen = white+opacity(1);

// LEGACY NAMES
// material[] graph_colors = surface_pens;
// material[] graph_colors2 = new material[] { graph_colors[0], graph_colors[3] };
// pen graph_meshpen = mesh_pen;
// pen aux_surface_meshpen = aux_mesh_pen;

settings.prc = false;

// Settings for debugging
settings.render=1;
int resx = 20;
int resy = 20;

// Settings for pdf -- they are toggled below
// settings.render = 16;
// settings.prc = false;
// int resx = 300; // Resolution of spline surfaces
// int resy = 300;

// Problem with graphic card; if maxtile=(0,0) works, use it
// Source: tex.stackexchange.com/questions/176539/
settings.maxtile=(600, 600); 

import "../asy/ma2712.asy" as ma2712;
\end{asydef}

% Do we want high resolution?
\newif\ifasyhighres
\asyhighresfalse
\input{asy_flags}

\ifasyhighres
\begin{asydef}
settings.render = 16;
int resx = 300;
int resy = 300;
\end{asydef}
\fi

%--------------------------------------------------------------------
%--- Math Notation
%--------------------------------------------------------------------
\newcommand{\al}{\alpha} 
\newcommand{\be}{\beta} 
\newcommand{\ga}{\gamma} 
\newcommand{\de}{\delta} 
\newcommand{\ep}{\varepsilon} 
\newcommand{\ze}{\zeta} 
\newcommand{\et}{\eta} 
\renewcommand{\th}{\theta} % Already defined by T1 encoding
\newcommand{\io}{\iota} 
\newcommand{\ka}{\kappa} 
\newcommand{\la}{\lambda} 
\newcommand{\rh}{\varrho} 
\renewcommand{\si}{\sigma} % Already defined by siunitx
\newcommand{\ta}{\tau} 
\newcommand{\ph}{\varphi} 
\newcommand{\ch}{\chi} 
\newcommand{\ps}{\psi} 
\newcommand{\om}{\omega} 
\newcommand{\Ga}{\Gamma} 
\newcommand{\De}{\Delta} 
\newcommand{\Th}{\Theta}
\newcommand{\La}{\Lambda} 
\newcommand{\Si}{\Sigma} 
\newcommand{\Ph}{\Phi} 
\newcommand{\Ps}{\Psi} 
\newcommand{\Om}{\Omega}
 
\def\inv{^{-1}} 
\def\x{\times}
\def\p{\partial} 
\def\N{{\mathbb N}}
\def\R{{\mathbb R}}
\def\exp{\operatorname{exp}}
\def\one{\mathbbm{1}}

\def\leq{\leqslant}
\def\geq{\geqslant}

\let\on=\operatorname
\let\wt=\widetilde
\let\wh=\widehat
\let\ol=\overline

\let\mb=\mathbb
\let\mc=\mathcal
\let\mf=\mathfrak

\newcommand{\ud}{\,\mathrm{d}}
\renewcommand{\vec}[1]{\bm{\mathrm{#1}}}
